%%% ======= Beamer ======
\documentclass[usenames,dvipsnames,t]{beamer}
% \documentclass[usenames,dvipsnames, handout]{beamer}
\beamertemplatenavigationsymbolsempty % remove toolbar at the bottom of slides
\usepackage{appendixnumberbeamer} % for appendix
\usetheme{Madrid}
\usecolortheme{default}
\useinnertheme{circles}

\usepackage{fontawesome}

% Define commands for social media icons with links
\newcommand{\linkedin}{\href{https://www.linkedin.com/in/ThibeauWouters}{\textcolor{black}{\faLinkedin}}}
\newcommand{\github}{\href{https://github.com/ThibeauWouters}{\textcolor{black}{\faGithub}}}
\newcommand{\myemail}{\href{mailto:t.r.i.wouters@uu.nl}{\textcolor{black}{\faEnvelope}}}

\newcommand{\ghlink}[1]{\href{https://github.com/#1}{\textcolor{black}{\faGithub}}}

\definecolor{customblue}{HTML}{7db8dc}
\newcommand{\thetaeos}{\boldsymbol{\theta}_{\rm{EOS}}}
\newcommand{\boldtheta}{\boldsymbol{\theta}}



\setbeamercolor{author in head/foot}{bg=blue!10, fg=blue}
\setbeamercolor{title in head/foot}{bg=blue!10, fg=blue}
\setbeamercolor{date in head/foot}{bg=blue!10, fg=blue}

\makeatletter
\setbeamertemplate{footline}{
  \leavevmode%
  \hbox{%
  \begin{beamercolorbox}[wd=.333333\paperwidth,ht=2.25ex,dp=1ex,center]{author in head/foot}%
    \usebeamerfont{author in head/foot}\insertshortauthor\expandafter\ifblank\expandafter{\beamer@shortinstitute}{}{~~(\insertshortinstitute)}
  \end{beamercolorbox}%
  \begin{beamercolorbox}[wd=.333333\paperwidth,ht=2.25ex,dp=1ex,center]{title in head/foot}%
    \usebeamerfont{title in head/foot}\insertshorttitle
  \end{beamercolorbox}%
  \begin{beamercolorbox}[wd=.333333\paperwidth,ht=2.25ex,dp=1ex,right]{date in head/foot}%
    \usebeamerfont{date in head/foot}\insertshortdate{}\hspace*{2em}
    \insertframenumber{}%
%     / \inserttotalframenumber
    \hspace*{2ex} 
  \end{beamercolorbox}}%
  \vskip0pt%
}
\makeatother

% Show the TOC at the beginning
\AtBeginSection[]{
  \addtocounter{framenumber}{-1}
  \begin{frame}[plain]
      \frametitle{Contents}
      \tableofcontents[currentsection,subsectionstyle=shaded/shaded/hide]
  \end{frame}
}

% Show TOC at beginning of each subsection
\AtBeginSubsection[]{
  \addtocounter{framenumber}{-1}
  \begin{frame}[plain]
      \frametitle{Contents}
      \tableofcontents[currentsection, currentsubsection, subsectionstyle=show/shaded/hide]
  \end{frame}
}


\colorlet{beamer@blendedblue}{blue!70} % change color theme

\usepackage[style=numeric-comp,sorting=none,backend=biber]{biblatex}%<- specify style
\addbibresource{references.bib}%<- specify bib file

\usepackage[inkscapearea=page]{svg}
\usepackage{adjustbox}


% For appendix
\newcommand{\backupbegin}{
   \newcounter{framenumberappendix}
   \setcounter{framenumberappendix}{\value{framenumber}}
}
\newcommand{\backupend}{
   \addtocounter{framenumberappendix}{-\value{framenumber}}
   \addtocounter{framenumber}{\value{framenumberappendix}} 
}

\setbeamertemplate{bibliography item}{\insertbiblabel} % improved references



% Other preamble stuff:
\usepackage{preamble}

%%% Uncomment for another color palette
% \definecolor{Logo1}{rgb}{0.0, 0, 0.7}
% \definecolor{Logo2}{rgb}{2.55, 2.55, 2.55}

% \setbeamercolor*{palette primary}{bg=Logo1, fg=white}
% \setbeamercolor*{palette secondary}{bg=Logo2, fg=white}
% \setbeamercolor*{palette tertiary}{bg=white, fg=Logo1}
% \setbeamercolor*{palette quaternary}{bg=white,fg=white}
% \setbeamercolor{structure}{fg=Logo1} % itemize, enumerate, etc
% \setbeamercolor{section in toc}{fg=Logo1} % TOC sections

% For figures
\usepackage{import}
\usepackage{xifthen}
\usepackage{pdfpages}
\usepackage{transparent}
\usepackage{mdframed}
\usepackage{subcaption}

\setbeamertemplate{caption}[numbered]



% --- Inkscape figures:
\newcommand{\incfig}[2][0.75\textwidth]{%
    \def\svgwidth{\columnwidth}
    \resizebox{#1}{!}{\import{Inkscape/}{#2.pdf_tex}}
}

% --- Height of frame
\newlength{\myheight}
\setlength{\myheight}{7cm}

\newlength\myheightfigureintext
\newlength\mydepthfigureintext
\settototalheight\myheightfigureintext{Xygp}
\settodepth\mydepthfigureintext{Xygp}
\setlength\fboxsep{0pt}

\usepackage{tikz}
\usepackage[absolute,overlay]{textpos} % for precise positioning

\usepackage{ifthen}
\newcommand{\showoverview}[1]{%
  \def\x{2mm}%
  \def\y{1mm}%
  
  Analyzing a multi-messenger \bns{\textbf{binary neutron star}} signal:
  \begin{enumerate}
    \vspace{\x}
    \item \ifthenelse{\equal{#1}{1}}{\textbf{Gravitational waves}}{Gravitational waves}
    
    % \vspace{\x}
    
    % \item \ifthenelse{\equal{#1}{2}}{\textbf{Glitch mitigation}}{Glitch mitigation}
    
    \vspace{\x}
    
    \item \ifthenelse{\equal{#1}{2}}{\textbf{Electromagnetic counterparts}}{Electromagnetic counterparts}
    
    \vspace{\x}
    
    \item \ifthenelse{\equal{#1}{3}}{\textbf{Nuclear equation of state}}{Nuclear equation of state}
  \end{enumerate}
  
  \vspace{\y}
  \centering
  \incfig[0.975\textwidth]{talk_overview}
}



%------------------------------------------------------------
%This block of code defines the information to appear in the
%Title page
\title[] %optional
{Encoding neutron star information into neural priors for gravitational wave analyses}

\author[Thibeau Wouters]{Thibeau Wouters \\ \vspace{2mm} \href{mailto:t.r.i.wouters@uu.nl}{t.r.i.wouters@uu.nl} \newline \github \quad \linkedin \quad \myemail}

\date{Be-NL GW meeting}

%End of title page configuration block
%------------------------------------------------------------



%------------------------------------------------------------
%The next block of commands puts the table of contents at the 
%beginning of each section and highlights the current section:

%------------------------------------------------------------


\begin{document}

{
\usebackgroundtemplate{\transparent{0.5}{\includegraphics[width=\paperwidth,height=\paperheight]{Figures/tintin_BNS_2.png}}}

\begin{frame}[plain, noframenumbering]

  \begin{tikzpicture}[remember picture,overlay]
    \node[fill=customblue, fill opacity=0.75, text opacity=1, rounded corners=10pt, inner sep=15pt] at (current page.center) {
      \begin{minipage}{0.8\textwidth}
        \centering
        \textbf{Encoding neutron star information into neural priors for gravitational wave analyses}\\[1.5ex]
        \normalsize Thibeau Wouters \\[0.5ex]
        \github \quad \linkedin \quad \myemail
      \end{minipage}
    };
  \end{tikzpicture}
  
  \vspace{7cm}

  \begin{columns}
  \column{0.35\textwidth}
  \begin{figure}
    \centering
    \vspace{1.5mm}
    \includegraphics[width=0.75\linewidth]{Figures/utrecht-university.png}
  \end{figure}
  \column{0.35\textwidth}
  \begin{figure}
    \centering
    \includegraphics[width=0.75\linewidth]{Figures/Nikhef_logo-transparent.png}
  \end{figure}
\end{columns}
  
  \end{frame}
}


\begin{frame}
\frametitle{Table of Contents}
\tableofcontents[hideallsubsections]
\end{frame}

\section{Key idea}

\begin{frame}{Key idea}
  \def\x{4mm}
  \def\y{1mm}

  \begin{itemize}
    \item GW parameter estimation: Bayesian inference
    \begin{align*}
      \mathcal{P}(\theta_{\rm{GW}} | d ) &\propto \mathcal{L}(d | \theta_{\rm{GW}}) \red{\pi(\theta_{\rm{GW}})} \\
      {\rm{posterior}} &\propto {\rm{likelihood}} \times {\rm{\red{prior}}}
    \end{align*}

    \item By default, we choose \red{agnostic priors} (e.g., uniform)

    \vspace{\x}
    \pause

    \item What if we \textbf{\textit{do}} have non-trivial prior information? % E.g. population, independent observations,...
  \end{itemize}

  \pause
  \vspace{\x}

  \begin{tcolorbox}[colback=blue!5!white,colframe=blue!75!black,title=Neural priors]
    Flexible way to encode information into priors
  \end{tcolorbox}
\end{frame}

\begin{frame}{Key idea}
  \def\x{3mm}
  \def\y{0.9mm}
  \def\z{0.5mm}

  \begin{enumerate}
    \item Get predictions for GW parameters (populations, physics,...)
    \begin{itemize}
      \item Can be an expensive/complicated model
    \end{itemize}

    \vspace{\y}
    
    \item Gives a dataset of samples
    
    \pause

    \vspace{\y}
    
    \item Emulate with normalizing flow: \jaxtwo{\textbf{neural priors}}
    \begin{itemize}
      \item Generative machine learning model

      \vspace{\z}

      \item Normalized

      \vspace{\z}

      \item Generate samples, evaluate density

      \vspace{\z}

      \item Accurate in high dimensions
    \end{itemize}

    \vspace{\y}

    \item Implemented in \textsc{bilby}: easy to use in GW parameter estimation
  \end{enumerate}

  \vspace{\x}
  \pause

  \begin{tcolorbox}[colback=blue!5!white,colframe=blue!75!black,title=Case study]
    Apply this to neutron stars
  \end{tcolorbox}

  (Do you have a use case? Let's talk!)
\end{frame}

\section{Case study: neutron star information}

% \begin{frame}{Neutron star masses}

%   Models of the mass: agnostic or based on measurements

%   \vspace{4mm}

%   \begin{figure}
%     \centering
%     \includegraphics[width=0.90\linewidth]{Figures/populations_overview.pdf}
%   \end{figure}
% \end{frame}

\begin{frame}{Tidal deformability}
  \def\x{5mm}

  \begin{itemize}
    \item Neutron stars are tidally deformed in a binary
    
    \vspace{\x}

    \item Quantified by \red{tidal deformability} $\Lambda$
    
    \vspace{\x}

    \item Depends on equation of state (EOS): allow for non-trivial prior information
  \end{itemize}

  \vspace{\x}

  \centering
  \incfig[1.0\textwidth]{tidal}
\end{frame}

\begin{frame}{Tidal deformability constraints}
  \begin{itemize}
    \item \textbf{Heavy pulsars}: must support $2\,M_{\odot}$ neutron stars
  \end{itemize}

  \vspace{0.5cm}
  \centering
  \incfig[0.975\textwidth]{mass-radius}
\end{frame}

\begin{frame}{Neural priors for neutron stars}
  \def\x{2mm}
  \def\y{3mm}

  \begin{itemize}
    \item Example: uniform population \& $M_{\rm{max}} > 2.0 M_\odot$
    
    \vspace{\x}
    
    \item Emulate with normalizing flow: \textbf{\jaxtwo{neural prior}}
    
    \vspace{\x}
    
    \item Can combine any population \& EOS constraints
  \end{itemize}

  \vspace{\y}

  \centering
  \incfig[0.975\textwidth]{NFprior}
\end{frame}

\begin{frame}{Tidal deformability constraints}

  \def\x{2mm}
  \def\y{2mm}

  \begin{itemize}
    \item \textbf{Heavy pulsars}: must support $2\,M_{\odot}$ neutron stars

    \vspace{\y}

    \item \textbf{Chiral EFT}: nuclear theory predictions
    
    \vspace{\y}

    \item \textbf{NICER}: mass-radius observations of neutron stars
  \end{itemize}

  \vspace{0.5cm}
  \centering
  \incfig[0.95\textwidth]{R14_table}
\end{frame}


\begin{frame}{Source classification}
  \def\x{2mm}

  \begin{itemize}
    \item Also construct a \textbf{\nsbh{neutron star-black hole (NSBH)}} prior

    \vspace{\x}
  
    \item Enable Bayesian model selection
  \end{itemize}

  \vspace{\x}

  \centering
  \incfig[0.95\textwidth]{bns_vs_nsbh}
\end{frame}


% \begin{frame}{Equation of state}

%   \def\x{3mm}
%   \def\y{1mm}

%   Neutron stars probe the high-density part of the equation of state (EOS) of dense nuclear matter~\cite{Koehn:2024set}

%   \vspace{\x}

%   \begin{figure}
%     \centering
%     \includegraphics[width=0.85\linewidth]{Figures/Koehn_EOS.jpg}
%   \end{figure}
% \end{frame}



% \begin{frame}{Neural priors}
%   \def\x{1mm}
%   \begin{itemize}
%     \item Masses can also be informed by populations
%     \begin{itemize}
%       \item Uniform (agnostic)
%       \item Gaussian
%       \item Double Gaussian
%     \end{itemize}

%     \vspace{\x}

%     \item (BNS/NSBH) $\times$ ($3$ populations) $\times$ ($3$ EOS constraints) = $18$ priors
    
%     \vspace{\x}

%     \item Emulate with normalizing flow: \textbf{\jaxtwo{neural priors}}
%   \end{itemize}

%   \pause

%   \begin{figure}
%     \centering
%     \includegraphics[width=0.99\linewidth]{Figures/neural_priors.jpg}
%   \end{figure}
% \end{frame}

% \begin{frame}{Application}

%   \begin{itemize}
%     \item Implemented in \textsc{bilby}

%     \item Consider three ``low-mass'' GWs
  

%   \begin{columns}
%       \begin{column}{0.30\textwidth}
%         \begin{itemize}
%           \item GW170817
%           \item GW190425
%           \item GW230529
%         \end{itemize}
%       \end{column}
      
%       \begin{column}{0.69\textwidth}
%         \begin{figure}
%           \centering
%           \includegraphics[width=0.75\linewidth]{Figures/GW230529_mass_diagram.jpg}
%         \end{figure}
%       \end{column}
%     \end{columns}
%     \end{itemize}

% \end{frame}

\section{Teaser: GW170817}

\begin{frame}{GW170817, Gaussian population}

  \begin{figure}
    \centering
    \includegraphics[width=0.65\linewidth]{Figures/GW170817_corner_gaussian_bns_PRESENTATION.pdf}
  \end{figure}
  
\end{frame}

\section{Conclusion}

\begin{frame}{Conclusion}
  \def\y{4mm}

  \begin{itemize}
    \item Parameter estimation often uses agnostic priors

    \vspace{\y}
    
    \item \jaxtwo{\textbf{Neural priors}}: flexible way to encode prior information 

    \vspace{\y}

    \item Proof of concept: neutron stars
    
    \begin{itemize}
      \item Bayesian source classification

      \item Informed parameter constraints
    \end{itemize}
    
    \vspace{\y}
    
    \item \textbf{Case study in mind? Let's talk!}
    \begin{itemize}
      \item More neutron star physics (temperature, ...)?
      \item Boson stars?
      \item Cosmology?
      \item Populations?
    \end{itemize}
  \end{itemize}
\end{frame}

{
\usebackgroundtemplate{\transparent{0.5}{\includegraphics[width=\paperwidth,height=\paperheight]{Figures/tintin_BNS_2.png}}}

\begin{frame}[plain, noframenumbering]

  \begin{tikzpicture}[remember picture,overlay]
    \node[fill=customblue, fill opacity=0.75, text opacity=1, rounded corners=10pt, inner sep=15pt] at ([yshift=2cm]current page.center) {
      \begin{minipage}{0.8\textwidth}
        \centering
        \textbf{Thanks for listening!}
      \end{minipage}
    };
  \end{tikzpicture}

  \end{frame}
}

\section{More results}

\begin{frame}[allowframebreaks]{References}
  \nocite{my_misc_entry}
  \nocite{Kuifje}
  \printbibliography
\end{frame}

\appendix 

\begin{frame}{GW170817 -- classification}

  Showing $\log_{10}$ Bayes factors: negative = less preferred

  \begin{itemize}
    \item Strongly prefer BNS over NSBH
    \item Gaussian population, EOS inconclusive
  \end{itemize}
  \begin{figure}
    \centering
    \includegraphics[width=0.50\linewidth]{Figures/GW170817_EOS_source_classification.jpg}
  \end{figure}
  
\end{frame}

\begin{frame}{GW170817 -- parameter constraints}

  \begin{itemize}
    \item More equal mass ratio $q\geq 0.9$
    \item $\tilde{\Lambda}$ bimodal, resolved by extra EOS information
  \end{itemize}

  \begin{figure}
    \centering
    \includegraphics[width=0.525\linewidth]{Figures/GW170817_corner_gaussian_bns_PRESENTATION.pdf}
  \end{figure}
\end{frame}


\begin{frame}{GW190425 -- classification}

  Showing $\log_{10}$ Bayes factors: negative = less preferred
  \begin{itemize}
    \item Prefer BNS over NSBH, but less conclusive
    \item Most consistent with uniform population
  \end{itemize}

  \begin{figure}
    \centering
    \includegraphics[width=0.50\linewidth]{Figures/GW190425_EOS_source_classification.jpg}
  \end{figure}
\end{frame}

\begin{frame}{GW190425 -- parameter constraints}

  \begin{itemize}
    \item Less equal masses ($q\leq 0.9$)
    \item Higher distances
  \end{itemize}

  \begin{figure}
    \centering
    \includegraphics[width=0.525\linewidth]{Figures/GW190425_corner_uniform_bns_PRESENTATION.pdf}
  \end{figure}
\end{frame}

\begin{frame}{GW230529 -- classification}

  Showing $\log_{10}$ Bayes factors: negative = less preferred
  \begin{itemize}
    \item Decisive evidence for NSBH over BNS
    \item Weak evidence for population or EOS (low SNR)
  \end{itemize}

  \begin{figure}
    \centering
    \includegraphics[width=0.50\linewidth]{Figures/GW230529_EOS_source_classification.jpg}
  \end{figure}
\end{frame}


\begin{frame}{GW230529 -- parameter constraints}

  \begin{itemize}
    \item Mass ratio more constrained $\rightarrow$ $\chi_{1z}$ more constrained
    \item Higher distances
  \end{itemize}

  \begin{figure}
    \centering
    \includegraphics[width=0.525\linewidth]{Figures/GW230529_corner_gaussian_nsbh_PRESENTATION.pdf}
  \end{figure}
\end{frame}

\end{document}