%%% ======= Beamer ======
\documentclass[usenames,dvipsnames,t]{beamer}
\beamertemplatenavigationsymbolsempty % remove toolbar at the bottom of slides
\usepackage{appendixnumberbeamer} % for appendix
\usetheme{Madrid}
\usecolortheme{default}
\useinnertheme{circles}

\usepackage{fontawesome}
\usepackage{colortbl}  % For \cellcolor in tables

% Define commands for social media icons with links
\newcommand{\linkedin}{\href{https://www.linkedin.com/in/ThibeauWouters}{\textcolor{black}{\faLinkedin}}}
\newcommand{\github}{\href{https://github.com/ThibeauWouters}{\textcolor{black}{\faGithub}}}
\newcommand{\myemail}{\href{mailto:t.r.i.wouters@uu.nl}{\textcolor{black}{\faEnvelope}}}

\newcommand{\ghlink}[1]{\href{https://github.com/#1}{\textcolor{black}{\faGithub}}}

\definecolor{customblue}{HTML}{7db8dc}
\newcommand{\thetaeos}{\boldsymbol{\theta}_{\rm{EOS}}}
\newcommand{\boldtheta}{\boldsymbol{\theta}}

% Color definitions for Jeffrey's scale interpretation
% Taken from part of "rocket" color scheme of Seaborn
\definecolor{jeffreysred1}{HTML}{f6cdb0}  % barely worth mentioning
\definecolor{jeffreysred2}{HTML}{fc9074}  % substantial
\definecolor{jeffreysred3}{HTML}{f4744c}  % strong
\definecolor{jeffreysred4}{HTML}{ef5b43}  % very strong
\definecolor{jeffreysred5}{HTML}{f52a44}  % decisive

\setbeamercolor{author in head/foot}{bg=blue!10, fg=blue}
\setbeamercolor{title in head/foot}{bg=blue!10, fg=blue}
\setbeamercolor{date in head/foot}{bg=blue!10, fg=blue}

\makeatletter
\setbeamertemplate{footline}{
  \leavevmode%
  \hbox{%
  \begin{beamercolorbox}[wd=.333333\paperwidth,ht=2.25ex,dp=1ex,center]{author in head/foot}%
    \usebeamerfont{author in head/foot}\insertshortauthor\expandafter\ifblank\expandafter{\beamer@shortinstitute}{}{~~(\insertshortinstitute)}
  \end{beamercolorbox}%
  \begin{beamercolorbox}[wd=.333333\paperwidth,ht=2.25ex,dp=1ex,center]{title in head/foot}%
    \usebeamerfont{title in head/foot}\insertshorttitle
  \end{beamercolorbox}%
  \begin{beamercolorbox}[wd=.333333\paperwidth,ht=2.25ex,dp=1ex,right]{date in head/foot}%
    \usebeamerfont{date in head/foot}\insertshortdate{}\hspace*{2em}
    \insertframenumber{}%
    \hspace*{2ex}
  \end{beamercolorbox}}%
  \vskip0pt%
}
\makeatother

% Show the TOC at the beginning of each section
\AtBeginSection[]{
  \addtocounter{framenumber}{-1}
  \begin{frame}[plain]
      \frametitle{Contents}
      \tableofcontents[currentsection,subsectionstyle=shaded/shaded/hide]
  \end{frame}
}

\colorlet{beamer@blendedblue}{blue!70} % change color theme

\usepackage[style=numeric-comp,sorting=none,backend=biber]{biblatex}%<- specify style
\addbibresource{references.bib}%<- specify bib file

\usepackage[inkscapearea=page]{svg}
\usepackage{adjustbox}

% For appendix
\newcommand{\backupbegin}{
   \newcounter{framenumberappendix}
   \setcounter{framenumberappendix}{\value{framenumber}}
}
\newcommand{\backupend}{
   \addtocounter{framenumberappendix}{-\value{framenumber}}
   \addtocounter{framenumber}{\value{framenumberappendix}}
}

\setbeamertemplate{bibliography item}{\insertbiblabel} % improved references

% Other preamble stuff:
\usepackage{preamble}

% For figures
\usepackage{import}
\usepackage{xifthen}
\usepackage{pdfpages}
\usepackage{transparent}
\usepackage{mdframed}
\usepackage{subcaption}

\setbeamertemplate{caption}[numbered]

\usepackage{multirow}

% --- Inkscape figures:
\newcommand{\incfig}[2][0.75\textwidth]{%
    \def\svgwidth{\columnwidth}
    \resizebox{#1}{!}{\import{Inkscape/}{#2.pdf_tex}}
}

% --- Height of frame
\newlength{\myheight}
\setlength{\myheight}{7cm}

\newlength\myheightfigureintext
\newlength\mydepthfigureintext
\settototalheight\myheightfigureintext{Xygp}
\settodepth\mydepthfigureintext{Xygp}
\setlength\fboxsep{0pt}

\usepackage{tikz}
\usepackage[absolute,overlay]{textpos} % for precise positioning

\usepackage{ifthen}

%------------------------------------------------------------
%This block of code defines the information to appear in the
%Title page
\title[GW231109 \& Neural priors]
{Analyzing GW231109\_235456 in the ET era and incorporating neutron star physics into future GW inference}

\author[Thibeau Wouters]{\textbf{Thibeau Wouters}, Anna Puecher, Peter T. H. Pang, Tim Dietrich, Chris Van Den Broeck \\ \vspace{2mm} \href{mailto:t.r.i.wouters@uu.nl}{t.r.i.wouters@uu.nl} \newline \github \quad \linkedin \quad \myemail}

\date[ET div 6]{ET Division 6 Meeting, \today}

%End of title page configuration block
%------------------------------------------------------------

\begin{document}

{
\usebackgroundtemplate{\transparent{0.5}{\includegraphics[width=\paperwidth,height=\paperheight]{Figures/tintin_BNS_2.png}}}

\begin{frame}[plain, noframenumbering]

  \begin{tikzpicture}[remember picture,overlay]
    \node[fill=customblue, fill opacity=0.75, text opacity=1, rounded corners=10pt, inner sep=15pt] at (current page.center) {
      \begin{minipage}{0.8\textwidth}
        \centering
        \textbf{Analyzing GW231109\_235456 in the ET era and incorporating neutron star physics into future GW inference}\\[1.5ex]

        \small \texttt{arXiv:2510.22290} \& \texttt{arXiv:2511.22987}\\[1.5ex]

        \small Thibeau Wouters, Anna Puecher, Peter T. H. Pang, Tim Dietrich, Chris Van Den Broeck \normalsize \\[0.5ex]
        \github \quad \linkedin \quad \myemail
      \end{minipage}
    };
  \end{tikzpicture}

  \vspace{7cm}

  \begin{columns}
  \column{0.35\textwidth}
  \begin{figure}
    \centering
    \vspace{1.5mm}
    \includegraphics[width=0.75\linewidth]{Figures/utrecht-university.png}
  \end{figure}
  \column{0.35\textwidth}
  \begin{figure}
    \centering
    \includegraphics[width=0.75\linewidth]{Figures/Nikhef_logo-transparent.png}
  \end{figure}
\end{columns}

  \end{frame}
}

\begin{frame}{Structure of this talk}
  \def\x{2mm}
  \def\y{1mm}

  Data analysis of neutron stars forms a \textbf{loop}:

  \vspace{\y}
  \begin{enumerate}
      \item \red{Constraining} the EOS with neutron star observations

      \vspace{\x}

      \item \red{Applying} EOS knowledge in neutron star data analysis (e.g., GW)
  \end{enumerate}

  \vspace{\x}

  How can we efficiently perform this loop?

  \vspace{\y}

  \centering
  \incfig[0.9\textwidth]{NS_to_EOS}
\end{frame}


\begin{frame}
\frametitle{Table of Contents}
\tableofcontents[hideallsubsections]
\end{frame}

\section{Part 1: Analyzing GW231109\_235456 in the ET era \small (\texttt{arXiv:2510.22290}) \normalsize}

\begin{frame}{GWTC-4.0 and GW231109\_235456}

  \def\x{3mm}

  \begin{itemize}
    \item<1-> GWTC-4.0 released recently~\cite{LIGOScientific:2025slb}:
    \begin{itemize}
      \item Over 200 gravitational wave events analyzed in total
    \end{itemize}

    \vspace{\x}

    \item<2-> \textbf{No confident binary neutron star detections}

    \vspace{\x}

    \item<3-> However: sub-threshold candidate \textbf{GW231109\_235456} identified~\cite{Niu:2025nha}
    \begin{itemize}
      \item \red{SNR $\sim 9.7$} (distance: $\sim 165$ Mpc)
      \item But mass closer to GW170817 than GW190425
    \end{itemize}
  \end{itemize}

  \vspace{5mm}

  \onslide<4->{
  \begin{tcolorbox}[colback=blue!10!white, colframe=blue!80!black, coltext=black]
  What can we learn about the EOS from such a merger?
  \end{tcolorbox}

  \vspace{5mm}

  \small (More on populations, remnant, EM counterpart: \texttt{arXiv:2510.22290}) \normalsize
  }

\end{frame}

\begin{frame}{GW231109\_235456: component masses}

  \def\x{2mm}

  Component masses compared to other low-mass GW events~\cite{LIGOScientific:2017vwq, LIGOScientific:2020aai, LIGOScientific:2024elc}

  \vspace{\x}

  \begin{figure}
    \centering
    \includegraphics[width=0.60\linewidth]{Figures/m1m2_overview.pdf}
  \end{figure}
\end{frame}

\begin{frame}{Parameter estimation on GW231109\_235456}
  \def\x{2mm}
  \begin{itemize}
    \item \texttt{IMRPhenomXP\_NRTidalv3}
    \item Standard priors for $m_i$, $\Lambda_i \leq 5000$, spins below $0.05$ or $0.4$
  \end{itemize}

  \begin{figure}
    \centering
    \includegraphics[width=0.525\linewidth]{Figures/comparison_l5000_spin.pdf}
  \end{figure}
\end{frame}

\begin{frame}{Constraining EOS from GW231109\_235456}

  \def\x{2mm}
  \def\z{1mm}

  \begin{itemize}
    \item Parametrized EOS inference: $26$ parameters in total
    \begin{itemize}
      \item Fixed crust
      \item Metamodel
      \item Speed-of-sound extension
    \end{itemize}

    \item Accelerate with GPUs: \textsc{jester}~\cite{Wouters:2025zju}: $\sim 1$ hour
  \end{itemize}

  \vspace{3mm}

  \centering
  \incfig[0.825\textwidth]{EOS_parametrizion}

\end{frame}

\begin{frame}{Constraining EOS from GW231109\_235456}

  \def\x{2mm}

  Constraints on radius of $1.4\,M_{\odot}$ neutron star ($R_{1.4}$):

  % \vspace{\x}

  \begin{figure}
    \centering
    \includegraphics[width=0.625\linewidth]{Figures/final_R14_histogram.pdf}
  \end{figure}

\end{frame}

\begin{frame}{Projection: Einstein Telescope \& Cosmic Explorer}

  \def\x{1mm}

  \begin{itemize}
    \item Simulate GW231109-like event with third-generation detectors

    \vspace{\x}

    \item Einstein Telescope: SNR $\sim 134$, with Cosmic Explorer: SNR $\sim 294$

    \vspace{\x}

    \item Recovery of tidal deformation improved
  \end{itemize}

  \vspace{\x}

  \begin{figure}
    \centering
    \includegraphics[width=0.99\linewidth]{Figures/anna_tim_favourite_plot_presentation.pdf}
  \end{figure}

\end{frame}

\begin{frame}{Projection: radius constraints}

  \def\x{2mm}

  Recover radius with accuracy of $300$-$400$ meters (ET+CE vs ET)

  \begin{figure}
    \centering
    \includegraphics[width=0.625\linewidth]{Figures/ET_full_injection_R14_histogram.pdf}
  \end{figure}
\end{frame}

\begin{frame}{Conclusion (part 1)}

  \def\x{3mm}

  \begin{itemize}
    \item GW231109\_235456: sub-threshold BNS candidate from O4a

    \vspace{\x}

    \item SNR matters for EOS inference
    \begin{itemize}
      \item Current detectors: poor constraints
      \item ET and CE: precise radius measurements ($\sim 300$-$400$ m)
    \end{itemize}

    \vspace{\x}

    \item \textsc{jester}: constrain EOS from 3G BNS in $\sim 1$ hour
  \end{itemize}

  \vspace{5mm}

  \centering
  \incfig[0.95\textwidth]{NS_to_EOS_part1}
  
\end{frame}

\section{Part 2: Neural priors for GW inference \small (\texttt{arXiv:2511.22987}) \normalsize}

\input{talk_structure_part2.tex}

\begin{frame}{Neural priors: motivation}
  \def\x{3mm}
  \def\y{1mm}

  \begin{itemize}
    \item Bayesian inference depends on choice of \red{priors}:
    \begin{equation*}
      \mathcal{P}(\theta_{\rm{GW}} | d ) \propto \frac{\mathcal{L}(d | \theta_{\rm{GW}}) \red{\pi(\theta_{\rm{GW}})}}{\mathcal{Z}}
    \end{equation*}

    \vspace{\x}

    \item By default, we use \red{agnostic priors}, but what if we \textit{do} have non-trivial prior information?

    \vspace{\x}

    \item Case study: neutron stars (NSs) and information from
    \begin{itemize}
      \item Population models
      \item Equation of state (EOS) constraints
    \end{itemize}
  \end{itemize}

  \vspace{\x}

  \begin{tcolorbox}[colback=blue!5!white,colframe=blue!75!black,title=Neural priors]
    Flexible way to encode NS physics into GW inference
  \end{tcolorbox}
\end{frame}

\begin{frame}{Neural priors: key idea}
  \def\x{3mm}

  Train normalizing flow (NF) on samples informed by populations and EOS
  $\rightarrow$ \jaxtwo{\textbf{neural prior}}

  \begin{figure}
    \centering
    \includegraphics[width=0.55\linewidth]{Figures/Figure1.pdf}
  \end{figure}
\end{frame}

\begin{frame}{NS population models}
  \def\x{3.5mm}

  Three fiducial population models for NS masses:

  \vspace{\x}

  \begin{enumerate}
    \item \textbf{Uniform}~\cite{LIGOScientific:2021qlt, Landry:2021hvl, Golomb:2024lds}:
    \begin{itemize}
      \item Only use EOS constraints for maximum mass ($\MTOV$)
      \item NS mass $\sim U[1\,\Msun, M_{\rm{TOV}}]$
    \end{itemize}

    \vspace{\x}

    \item \textbf{Gaussian}~\cite{Ozel:2016oaf}:
    \begin{itemize}
      \item NS mass $\sim \mathcal{N}(1.33\,\Msun, (0.09\,\Msun)^2)$
    \end{itemize}

    \vspace{\x}

    \item \textbf{Double Gaussian}~\cite{Alsing:2017bbc, Shao:2020bzt}:
    \begin{itemize}
      \item Weighted mixture of two Gaussians
      \item $0.65 \times \mathcal{N}(1.34\,\Msun, (0.07\,\Msun)^2) + 0.35 \times \mathcal{N}(1.80\,\Msun, (0.21\,\Msun)^2)$
    \end{itemize}
  \end{enumerate}

  \vspace{\x}

  For NSBH systems:
  \begin{itemize}
    \item BH mass $m_1^{\rm{src}}$: from $[\MTOV, 5\,\Msun]$
    \item NS mass $m_2^{\rm{src}}$: above models
  \end{itemize}
\end{frame}

\begin{frame}{EOS constraints}

  \def\x{2mm}
  \def\y{2mm}

  \begin{itemize}
    \item We use three EOS constraints~\cite{Koehn:2024set}:

    \begin{enumerate}
      \vspace{\y}
      \item \textbf{Heavy pulsars}: must support $2\,M_{\odot}$ NSs

      \vspace{\y}

      \item \textbf{Chiral EFT} ($\chi_{\rm{EFT}}$): nuclear theory predictions (softer EOS)

      \vspace{\y}

      \item \textbf{NICER}: mass-radius observations of NSs (stiffer EOS)
    \end{enumerate}

    \vspace{\x}

    \item Posterior samples obtained with \textsc{jester}~\cite{Wouters:2025zju}~\ghlink{nuclear-multimessenger-astronomy/jester}
  \end{itemize}

  \vspace{-1mm}

  \centering
  \incfig[0.95\textwidth]{R14_table}
\end{frame}

\begin{frame}{Normalizing flows}
  \def\x{3mm}

  Normalizing flows~\cite{Kobyzev:2019ydm, Papamakarios:2019fms}

  \begin{itemize}
    \item Neural density estimators: trainable bijections
    \item Often used in GW: \textsc{dingo}~\cite{Dax:2021tsq, Dax:2022pxd}, \textsc{nessai}~\cite{Williams:2021qyt,Williams:2023ppp}
    \item Generate samples, evaluate density: can be used as priors~\cite{Alsing:2021wef}
  \end{itemize}

  % \vspace{\x}

  \centering
  \incfig[0.85\textwidth]{NF}
\end{frame}

\begin{frame}{Construction of neural priors}
  \def\x{4mm}
  \def\y{2mm}

  Steps to generate training data:
  \begin{enumerate}
    \vspace{\y}

    \item Draw EOS posterior curve: determines $\MTOV$, $\Lambda(m)$

    \vspace{\y}

    \item Draw masses from population model

    \vspace{\y}

    \item Compute $\Lambda_i = \Lambda(m_i)$ for NSs (NSBH: $\Lambda_1 = 0$)
  \end{enumerate}

  \vspace{\x}

  Implementation:
  \begin{itemize}
    \vspace{\y}

    \item Created with \textsc{glasflow}~\cite{williams_uofgravityglasflow_2024, nflows}

    \vspace{\y}

    \item \texttt{CouplingNSF} architecture (neural spline flows~\cite{Durkan:2019nsq})

    \vspace{\y}

    \item Use as a \texttt{JointPrior} in \textsc{bilby} (\texttt{NFPrior})
    \begin{itemize}
      \item Sample \& logpdf: evaluate NF
      \item Rescale: unit hypercube $\rightarrow$ multivariate Gaussian $\xrightarrow{\rm{NF}}$ data space
    \end{itemize}
  \end{itemize}
\end{frame}

\begin{frame}{All neural priors}
  \vspace{-1mm}

  \begin{figure}
    \centering
    \includegraphics[width=0.99\linewidth]{Figures/bns_nsbh_all_populations_chirp_tilde.pdf}
  \end{figure}
\end{frame}

\begin{frame}{Setup}

  \def\x{3mm}

  Analyze GW170817, GW190425, GW230529 with:
  \begin{itemize}
    \item \texttt{IMRPhenomXP\_NRTidalv3}

    \item Neural priors for $m_i$, $\Lambda_i$ (standard priors for other parameters)
  \end{itemize}

  \vspace{\x}

  Two contributions:
  \begin{enumerate}
    \item Model selection with Bayes factors
    \item Narrower constraints with neural priors
  \end{enumerate}

  \vspace{\x}

  \begin{table}
    \centering
    \caption*{\textbf{Jeffreys' scale for Bayes factors} ($\log_{10}$ scale)}
    \small
    \begin{tabular}{ccc}
      \hline
      $\log_{10}(\mathcal{B}_1^2)$ & Interpretation & Color \\
      \hline
      $[0,\, \tfrac{1}{2}]$ & Barely worth mentioning & \cellcolor{jeffreysred1} \\
      $[\tfrac{1}{2},\, 1]$ & Substantial & \cellcolor{jeffreysred2} \\
      $[1,\, \tfrac{3}{2}]$ & Strong & \cellcolor{jeffreysred3} \\
      $[\tfrac{3}{2},\, 2]$ & Very strong & \cellcolor{jeffreysred4} \\
      $> 2$ & Decisive & \cellcolor{jeffreysred5} \\
      \hline
    \end{tabular}
  \end{table}

\end{frame}

\begin{frame}{Source classification: All events}

  $\log_{10}$ Bayes factors relative to model with highest evidence (`ref.')

  \vspace{1mm}

  \begin{table}
    \centering
    \scriptsize
    \begin{tabular}{|l|l|l|c|c|c|}
    \hline
    \textbf{Source} & \textbf{Population} & \textbf{EOS} & \textbf{GW170817} & \textbf{GW190425} & \textbf{GW230529} \\
    \hline\hline
    \multirow{9}{*}{BNS} & \multirow{3}{*}{Uniform} & PSRs & \cellcolor{jeffreysred4}$-1.83$ & \cellcolor{jeffreysred1}$-0.07$ & \cellcolor{jeffreysred5}$-13.14$ \\
    \cline{3-6}
     &  & PSRs+$\chi_{\rm{EFT}}$ & \cellcolor{jeffreysred2}$-0.80$ & \cellcolor{jeffreysred1}$-0.11$ & \cellcolor{jeffreysred5}$-13.12$ \\
    \cline{3-6}
     &  & PSRs+NICER & \cellcolor{jeffreysred4}$-1.58$ & \textbf{ref.} & \cellcolor{jeffreysred5}$-12.92$ \\
    \cline{2-6}
     & \multirow{3}{*}{Gaussian} & PSRs & \cellcolor{jeffreysred2}$-0.68$ & \cellcolor{jeffreysred5}$-6.89$ & \cellcolor{jeffreysred5}$-18.82$ \\
    \cline{3-6}
     &  & PSRs+$\chi_{\rm{EFT}}$ & \textbf{ref.} & \cellcolor{jeffreysred5}$-8.47$ & \cellcolor{jeffreysred5}$-18.83$ \\
    \cline{3-6}
     &  & PSRs+NICER & \cellcolor{jeffreysred2}$-0.76$ & \cellcolor{jeffreysred5}$-5.45$ & \cellcolor{jeffreysred5}$-18.81$ \\
    \cline{2-6}
     & \multirow{3}{*}{Double Gaussian} & PSRs & \cellcolor{jeffreysred3}$-1.36$ & \cellcolor{jeffreysred2}$-0.55$ & \cellcolor{jeffreysred5}$-13.75$ \\
    \cline{3-6}
     &  & PSRs+$\chi_{\rm{EFT}}$ & \cellcolor{jeffreysred2}$-0.59$ & \cellcolor{jeffreysred2}$-0.79$ & \cellcolor{jeffreysred5}$-13.77$ \\
    \cline{3-6}
     &  & PSRs+NICER & \cellcolor{jeffreysred2}$-0.92$ & \cellcolor{jeffreysred2}$-0.57$ & \cellcolor{jeffreysred5}$-13.71$ \\
    \hline\hline
    \multirow{9}{*}{NSBH} & \multirow{3}{*}{Uniform} & PSRs & \cellcolor{jeffreysred5}$-224.65$ & \cellcolor{jeffreysred4}$-1.52$ & \cellcolor{jeffreysred1}$-0.08$ \\
    \cline{3-6}
     &  & PSRs+$\chi_{\rm{EFT}}$ & \cellcolor{jeffreysred5}$-224.66$ & \cellcolor{jeffreysred3}$-1.35$ & \cellcolor{jeffreysred1}$-0.02$ \\
    \cline{3-6}
     &  & PSRs+NICER & \cellcolor{jeffreysred5}$-224.66$ & \cellcolor{jeffreysred4}$-1.63$ & \cellcolor{jeffreysred1}$-0.25$ \\
    \cline{2-6}
     & \multirow{3}{*}{Gaussian} & PSRs & \cellcolor{jeffreysred5}$-224.67$ & \cellcolor{jeffreysred2}$-0.82$ & \cellcolor{jeffreysred1}$-0.05$ \\
    \cline{3-6}
     &  & PSRs+$\chi_{\rm{EFT}}$ & \cellcolor{jeffreysred5}$-224.66$ & \cellcolor{jeffreysred3}$-1.11$ & \cellcolor{jeffreysred1}$-0.20$ \\
    \cline{3-6}
     &  & PSRs+NICER & \cellcolor{jeffreysred5}$-224.66$ & \cellcolor{jeffreysred3}$-1.43$ & \textbf{ref.} \\
    \cline{2-6}
     & \multirow{3}{*}{Double Gaussian} & PSRs & \cellcolor{jeffreysred5}$-224.67$ & \cellcolor{jeffreysred5}$-4.11$ & \cellcolor{jeffreysred1}$-0.14$ \\
    \cline{3-6}
     &  & PSRs+$\chi_{\rm{EFT}}$ & \cellcolor{jeffreysred5}$-224.68$ & \cellcolor{jeffreysred5}$-3.83$ & \cellcolor{jeffreysred1}$-0.13$ \\
    \cline{3-6}
     &  & PSRs+NICER & \cellcolor{jeffreysred5}$-224.67$ & \cellcolor{jeffreysred5}$-24.31$ & \cellcolor{jeffreysred1}$-0.05$ \\
    \hline
    \end{tabular}
  \end{table}
\end{frame}

\begin{frame}{Parameter constraints: GW170817 (Gaussian population)}

  \vspace{-3mm}
  \begin{figure}
    \centering
    \includegraphics[width=0.69\linewidth]{Figures/GW170817_corner_gaussian_bns.pdf}
  \end{figure}
\end{frame}

\begin{frame}{Parameter constraints: GW190425 (uniform population)}

  \vspace{-3mm}

  \begin{figure}
    \centering
    \includegraphics[width=0.69\linewidth]{Figures/GW190425_corner_uniform_bns.pdf}
  \end{figure}
\end{frame}

\begin{frame}{Parameter constraints: GW230529 (Gaussian population)}

  \vspace{-3mm}

  \begin{figure}
    \centering
    \includegraphics[width=0.68\linewidth]{Figures/GW230529_corner_gaussian_nsbh.pdf}
  \end{figure}
\end{frame}


\begin{frame}{Discussion: Parameter constraints}
  \def\x{2mm}

  \textbf{GW170817:}
  \begin{itemize}
    \item More equal mass ratio: $q \geq 0.9$
    \item Constrained tidal deformability $\tilde{\Lambda}$
    \item Higher luminosity distance compared to agnostic prior
    \item Agrees with multimessenger analyses~\cite{Radice:2018ozg, Coughlin:2018fis, LIGOScientific:2018hze, Pang:2022rzc, Breschi:2024qlc}
  \end{itemize}

  \vspace{\x}

  \textbf{GW190425:}
  \begin{itemize}
    \item Less equal masses: $q \leq 0.9$; $\tilde{\Lambda} \approx 200$ (prior-dominated, low SNR)
    \item Higher luminosity distance: $182^{+41}_{-49}$ Mpc vs. $157^{+64}_{-65}$ Mpc
  \end{itemize}

  \vspace{\x}

  \textbf{GW230529:}
  \begin{itemize}
    \item Mass ratio more constrained: $q \leq 0.4$ (improved spin constraints)
    \item Tidal posteriors dominated by priors
    \item Higher luminosity distance: $235^{+59}_{-58}$ Mpc vs. $201^{+84}_{-97}$ Mpc
  \end{itemize}
\end{frame}



\begin{frame}{Conclusion (part 2)}
  \def\x{3mm}
  \def\y{1mm}
  \def\z{5mm}

  \begin{itemize}
    \item  \jaxtwo{\textbf{Neural priors}}: Flexible way to encode non-trivial prior information

    \vspace{\x}

    \item Two highlights:
    \vspace{\y}
    \begin{enumerate}
      \item Bayesian model selection

      \vspace{\y}

      \item Informed parameter constraints
    \end{enumerate}

    \vspace{\x}

    \item Implemented in \textsc{bilby}

    \vspace{\x}

    \item Data-driven approach: easy to extend/generalize

    \vspace{\x}

    \item Future work: apply to 3G BNS
  \end{itemize}
\end{frame}

\section{Conclusion}

\begin{frame}{Conclusion}
  \def\x{3mm}
  \def\y{2mm}
  \def\z{5mm}

  Methods for closing the neutron star data analysis loop:
  \vspace{\y}
  \begin{enumerate}
    \item \textsc{jester}: from NS observations to EOS constraints
    
    \vspace{\y}

    \item Neural priors: incorporate EOS knowledge into GW inference
  \end{enumerate}

  \vspace{\x}

  \centering
  \incfig[0.9\textwidth]{NS_to_EOS_end}
\end{frame}

\begin{frame}[allowframebreaks]{References}
  \printbibliography
\end{frame}

\appendix

\backupbegin

\begin{frame}{Posterior distributions for ET/ET+CE injections}

  \def\x{2mm}

  \begin{figure}
    \centering
    \includegraphics[width=0.60\linewidth]{Figures/comparison_new_ET_vs_ET_CE.pdf}
  \end{figure}

\end{frame}

\backupend

\end{document}
