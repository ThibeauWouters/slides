%%% ======= Beamer ======
\documentclass[usenames,dvipsnames,t]{beamer}
\beamertemplatenavigationsymbolsempty % remove toolbar at the bottom of slides
\usepackage{appendixnumberbeamer} % for appendix
\usetheme{Madrid}
\usecolortheme{default}
\useinnertheme{circles}

\usepackage{fontawesome}

% Define commands for social media icons with links
\newcommand{\linkedin}{\href{https://www.linkedin.com/in/ThibeauWouters}{\textcolor{black}{\faLinkedin}}}
\newcommand{\github}{\href{https://github.com/ThibeauWouters}{\textcolor{black}{\faGithub}}}
\newcommand{\myemail}{\href{mailto:t.r.i.wouters@uu.nl}{\textcolor{black}{\faEnvelope}}}

\newcommand{\ghlink}[1]{\href{https://github.com/#1}{\textcolor{black}{\faGithub}}}

\definecolor{customblue}{HTML}{7db8dc}
\newcommand{\thetaeos}{\boldsymbol{\theta}_{\rm{EOS}}}
\newcommand{\boldtheta}{\boldsymbol{\theta}}

\setbeamercolor{author in head/foot}{bg=blue!10, fg=blue}
\setbeamercolor{title in head/foot}{bg=blue!10, fg=blue}
\setbeamercolor{date in head/foot}{bg=blue!10, fg=blue}

\makeatletter
\setbeamertemplate{footline}{
  \leavevmode%
  \hbox{%
  \begin{beamercolorbox}[wd=.333333\paperwidth,ht=2.25ex,dp=1ex,center]{author in head/foot}%
    \usebeamerfont{author in head/foot}\insertshortauthor\expandafter\ifblank\expandafter{\beamer@shortinstitute}{}{~~(\insertshortinstitute)}
  \end{beamercolorbox}%
  \begin{beamercolorbox}[wd=.333333\paperwidth,ht=2.25ex,dp=1ex,center]{title in head/foot}%
    \usebeamerfont{title in head/foot}\insertshorttitle
  \end{beamercolorbox}%
  \begin{beamercolorbox}[wd=.333333\paperwidth,ht=2.25ex,dp=1ex,right]{date in head/foot}%
    \usebeamerfont{date in head/foot}\insertshortdate{}\hspace*{2em}
    \insertframenumber{}%
    \hspace*{2ex}
  \end{beamercolorbox}}%
  \vskip0pt%
}
\makeatother

% Show the TOC at the beginning of each section
\AtBeginSection[]{
  \addtocounter{framenumber}{-1}
  \begin{frame}[plain]
      \frametitle{Contents}
      \tableofcontents[currentsection,subsectionstyle=shaded/shaded/hide]
  \end{frame}
}

\colorlet{beamer@blendedblue}{blue!70} % change color theme

\usepackage[style=numeric-comp,sorting=none,backend=biber]{biblatex}%<- specify style
\addbibresource{references.bib}%<- specify bib file

\usepackage[inkscapearea=page]{svg}
\usepackage{adjustbox}

% For appendix
\newcommand{\backupbegin}{
   \newcounter{framenumberappendix}
   \setcounter{framenumberappendix}{\value{framenumber}}
}
\newcommand{\backupend}{
   \addtocounter{framenumberappendix}{-\value{framenumber}}
   \addtocounter{framenumber}{\value{framenumberappendix}}
}

\setbeamertemplate{bibliography item}{\insertbiblabel} % improved references

% Other preamble stuff:
\usepackage{preamble}

% For figures
\usepackage{import}
\usepackage{xifthen}
\usepackage{pdfpages}
\usepackage{transparent}
\usepackage{mdframed}
\usepackage{subcaption}

\setbeamertemplate{caption}[numbered]

% --- Inkscape figures:
\newcommand{\incfig}[2][0.75\textwidth]{%
    \def\svgwidth{\columnwidth}
    \resizebox{#1}{!}{\import{Inkscape/}{#2.pdf_tex}}
}

% --- Height of frame
\newlength{\myheight}
\setlength{\myheight}{7cm}

\newlength\myheightfigureintext
\newlength\mydepthfigureintext
\settototalheight\myheightfigureintext{Xygp}
\settodepth\mydepthfigureintext{Xygp}
\setlength\fboxsep{0pt}

\usepackage{tikz}
\usepackage[absolute,overlay]{textpos} % for precise positioning

%------------------------------------------------------------
%This block of code defines the information to appear in the
%Title page
\title[Analyzing GW231109\_235456] %optional
{Analyzing GW231109\_235456 and its implications for the neutron star equation of state}

\subtitle{NNV meeting}

\author[T. Wouters et al.]{Thibeau Wouters, Anna Puecher, Peter T. H. Pang, Tim Dietrich}

\date{\today}

%End of title page configuration block
%------------------------------------------------------------

\begin{document}

{
\usebackgroundtemplate{\transparent{0.5}{\includegraphics[width=\paperwidth,height=\paperheight]{/Users/Woute029/Documents/Code/slides/2025/AEI/Figures/tintin_BNS_2.png}}}

\begin{frame}[plain, noframenumbering]

  \begin{tikzpicture}[remember picture,overlay]
    \node[fill=customblue, fill opacity=0.75, text opacity=1, rounded corners=10pt, inner sep=15pt] at (current page.center) {
      \begin{minipage}{0.8\textwidth}
        \centering
        \textbf{Analyzing GW231109\_235456 and its implications for the neutron star equation of state}\\[1ex]
        \small Thibeau Wouters, Anna Puecher, Peter T. H. Pang, Tim Dietrich
      \end{minipage}
    };
  \end{tikzpicture}

  \end{frame}
}

\begin{frame}
\frametitle{Table of Contents}
\tableofcontents[hideallsubsections]
\end{frame}

\section{Introduction}

\begin{frame}{Neutron stars and the equation of state}

  \def\x{3mm}

  Neutron stars probe the high-density part of the equation of state (EOS) of dense nuclear matter~\cite{Koehn:2024set}

  \vspace{\x}

  \begin{figure}
    \centering
    \includegraphics[width=0.85\linewidth]{Figures/Koehn_EOS.jpg}
  \end{figure}
\end{frame}

\begin{frame}{Tidal deformability}
  \def\x{4mm}

  \begin{itemize}
    \item Neutron stars are tidally deformed in a binary
    
    \vspace{\x}

    \item Quantified by tidal deformability $\Lambda$, depends on equation of state

    \vspace{\x}

    \item \textbf{Challenge}: $\Lambda$ harder to measure than masses (higher-order effect)
  \end{itemize}

  \vspace{\x}

  \centering
  \incfig[1.0\textwidth]{tidal}
\end{frame}

\begin{frame}{Binary neutron star mergers: GW170817 and GW190425}

  \def\x{3mm}

  So far, two confident BNS detections:

  \vspace{\x}

  \begin{enumerate}
    \item<1-> \textbf{GW170817}~\cite{LIGOScientific:2017vwq}
    \begin{itemize}
      \item First multimessenger detection (GW + EM)
      \item SNR $\sim 32$ (distance: $\sim 40$ Mpc)
      \item Loud signal $\rightarrow$ excellent constraints on EOS
    \end{itemize}

    \vspace{\x}

    \item<2-> \textbf{GW190425}~\cite{LIGOScientific:2020aai}
    \begin{itemize}
      \item Heavier system: total mass $\sim 3.4\,M_{\odot}$
      \item SNR $\sim 12$ (distance: $\sim 160$ Mpc)
      \item Fainter, more massive $\rightarrow$ poor EOS constraints
    \end{itemize}
  \end{enumerate}

\end{frame}

\begin{frame}{GWTC-4.0 and GW231109\_235456}

  \def\x{3mm}

  \begin{itemize}
    \item<1-> GWTC-4.0 released recently~\cite{LIGOScientific:2025slb}:
    \begin{itemize}
      \item Over 200 gravitational wave detections in total
    \end{itemize}

    \vspace{\x}

    \item<2-> \textbf{No confident BNS detections} in O4a

    \vspace{\x}

    \item<3-> However: sub-threshold candidate \textbf{GW231109\_235456} identified~\cite{Niu:2025nha}
    \begin{itemize}
      \item \red{SNR $\sim 9.7$} (distance: $\sim 165$ Mpc)
      \item Fainter, but mass closer to GW170817 than GW190425
    \end{itemize}
  \end{itemize}

  \vspace{\x}

  \onslide<4->{
  \begin{tcolorbox}[colback=blue!10!white, colframe=blue!80!black, coltext=black]
  Can we still learn something about the EOS?
  \end{tcolorbox}
  }
\end{frame}

\begin{frame}{GW231109: component masses}

  \def\x{2mm}

  Component masses compared to other low-mass GW events~\cite{LIGOScientific:2017vwq, LIGOScientific:2020aai, LIGOScientific:2024elc}

  \vspace{\x}

  \begin{figure}
    \centering
    \includegraphics[width=0.60\linewidth]{Figures/m1m2_overview.pdf}
  \end{figure}
\end{frame}

% \begin{frame}{GW231109: source properties}

%   \def\x{2mm}

%   Parameter estimation with advanced waveform models~\cite{Abac:2023ujg, Colleoni:2023ple}

%   \vspace{\x}

%   \begin{figure}
%     \centering
%     \includegraphics[width=0.75\linewidth]{Figures/comparison_l5000_spin.pdf}
%   \end{figure}

%   \begin{itemize}
%     \item Low SNR $\rightarrow$ posteriors dominated by priors
%     \item Spin-mass ratio correlation
%   \end{itemize}
% \end{frame}

\begin{frame}{This work}

  \def\x{3mm}

  We analyze GW231109\_235456 and investigate:

  \vspace{\x}

  \begin{enumerate}
    \item Can we extract \textit{any} additional information about the EOS?

    \vspace{\x}

    \item How will future detectors (Einstein Telescope, Cosmic Explorer) improve constraints for similar events?
  \end{enumerate}

\end{frame}

\section{Methods}

\begin{frame}{Equation of state inference}
  \def\x{2mm}
  \def\y{5mm}

  \begin{itemize}
    \item To predict neutron star properties, we solve the TOV equations: ordinary differential equations (ODEs)

    \vspace{\x}

    \item<2-> Done while sampling parametrization for the EOS: \red{costly likelihood}
    
    \vspace{\x}

    \item<3-> Typically $\mathcal{O}(10^6)$ samples for inference

    \vspace{\x}

    \item<3-> Would take \red{days} with CPUs
  \end{itemize}

  \vspace{\y}

  \only<1>{
    \centering
    \incfig[0.95\textwidth]{TOV}
  }

  \only<2->{
    \centering
    \incfig[0.95\textwidth]{NS_likelihood}
  }

\end{frame}

\begin{frame}{GPU acceleration: \textsc{Jester}}
  \def\x{2mm}

  \begin{itemize}
    \item \textsc{Jester}~\ghlink{nuclear-multimessenger-astronomy/jester}~\cite{Wouters:2025zju}: \textsc{jax}-based TOV solver
    \begin{itemize}
      \item $1000\times$ faster, without compromises
      \item Full inference in \red{$\sim$1 hour on GPU} (vs days on CPU)
    \end{itemize}

    \vspace{\x}

    \item Enables systematic studies in EOS inference

  \end{itemize}

  \vspace{\x}

  \begin{figure}
    \centering
    \includegraphics[width=0.625\linewidth]{/Users/Woute029/Documents/Code/slides/2025/AEI/Figures/scaling_plot.pdf}
  \end{figure}
\end{frame}

\section{Results}

\begin{frame}{EOS constraints from GW231109}

  \def\x{2mm}

  Constraints on radius of $1.4\,M_{\odot}$ neutron star ($R_{1.4}$):

  % \vspace{\x}

  \begin{figure}
    \centering
    \includegraphics[width=0.625\linewidth]{Figures/final_R14_histogram.pdf}
  \end{figure}

\end{frame}

\begin{frame}{Projection: Einstein Telescope \& Cosmic Explorer}

  \def\x{1mm}

  \begin{itemize}
    \item Simulate GW231109-like event with third-generation detectors

    \vspace{\x}

    \item ET: SNR $\sim 134$, ET+CE network: SNR $\sim 294$

    \vspace{\x}

    \item Recovery improved
  \end{itemize}

  \vspace{\x}

  \begin{figure}
    \centering
    \includegraphics[width=0.99\linewidth]{Figures/anna_tim_favourite_plot_presentation.pdf}
  \end{figure}
  
\end{frame}


\begin{frame}{Projection: radius constraints}

  \def\x{2mm}

  Recover radius with accuracy of $300$-$400$ meters (ET+CE vs ET)

  \begin{figure}
    \centering
    \includegraphics[width=0.625\linewidth]{Figures/ET_full_injection_R14_histogram.pdf}
  \end{figure}
\end{frame}

\section{Conclusions}

\begin{frame}{Conclusions}

  \def\x{5mm}
  \def\z{5mm}

  \begin{itemize}
    \item GW231109\_235456: sub-threshold BNS candidate in O4a

    \vspace{\x}

    \item Low SNR ($\sim 9.7$): poor EOS constraints
    

    \vspace{\x}

    \item Future detectors (ET, CE) will dramatically improve constraints
    \begin{itemize}
      \item $R_{1.4}$ uncertainty: $\sim 300-400$ meters
    \end{itemize}

    \vspace{\x}

    \item GPU-accelerated tools (\textsc{Jester}):
    \begin{itemize}
      \item Enable fast systematic studies
      \item Can handle Einstein Telescope and Cosmic Explorer signals
    \end{itemize}
  \end{itemize}

  \vspace{\z}

  \textbf{Thank you for your attention!}
\end{frame}

\begin{frame}[allowframebreaks]{References}
  \printbibliography
\end{frame}

\appendix

\backupbegin

\begin{frame}{Posterior distributions for ET/ET+CE injections}

  \def\x{2mm}

  \begin{figure}
    \centering
    \includegraphics[width=0.60\linewidth]{Figures/comparison_new_ET_vs_ET_CE.pdf}
  \end{figure}
  
\end{frame}

\backupend

\end{document}
