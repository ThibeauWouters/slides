\documentclass[usenames,dvipsnames,t]{beamer}
% \documentclass[usenames,dvipsnames, handout]{beamer}
\beamertemplatenavigationsymbolsempty % remove toolbar at the bottom of slides
\usepackage{appendixnumberbeamer} % for appendix
\usetheme{Madrid}
\usecolortheme{default}
\useinnertheme{circles}

\usepackage{fontawesome}
\usepackage{colortbl}  % For \cellcolor in tables

% Define commands for social media icons with links
\newcommand{\linkedin}{\href{https://www.linkedin.com/in/ThibeauWouters}{\textcolor{black}{\faLinkedin}}}
\newcommand{\github}{\href{https://github.com/ThibeauWouters}{\textcolor{black}{\faGithub}}}
\newcommand{\myemail}{\href{mailto:t.r.i.wouters@uu.nl}{\textcolor{black}{\faEnvelope}}}

\newcommand{\ghlink}[1]{\href{https://github.com/#1}{\textcolor{black}{\faGithub}}}

\definecolor{customblue}{HTML}{7db8dc}
\newcommand{\thetaeos}{\boldsymbol{\theta}_{\rm{EOS}}}
\newcommand{\boldtheta}{\boldsymbol{\theta}}

% Color definitions for Jeffrey's scale interpretation
% Taken from part of "rocket" color scheme of Seaborn
\definecolor{jeffreysred1}{HTML}{f6cdb0}  % barely worth mentioning
\definecolor{jeffreysred2}{HTML}{fc9074}  % substantial
\definecolor{jeffreysred3}{HTML}{f4744c}  % strong
\definecolor{jeffreysred4}{HTML}{ef5b43}  % very strong
\definecolor{jeffreysred5}{HTML}{f52a44}  % decisive



\setbeamercolor{author in head/foot}{bg=blue!10, fg=blue}
\setbeamercolor{title in head/foot}{bg=blue!10, fg=blue}
\setbeamercolor{date in head/foot}{bg=blue!10, fg=blue}

\makeatletter
\setbeamertemplate{footline}{
  \leavevmode%
  \hbox{%
  \begin{beamercolorbox}[wd=.333333\paperwidth,ht=2.25ex,dp=1ex,center]{author in head/foot}%
    \usebeamerfont{author in head/foot}\insertshortauthor\expandafter\ifblank\expandafter{\beamer@shortinstitute}{}{~~(\insertshortinstitute)}
  \end{beamercolorbox}%
  \begin{beamercolorbox}[wd=.333333\paperwidth,ht=2.25ex,dp=1ex,center]{title in head/foot}%
    \usebeamerfont{title in head/foot}\insertshorttitle
  \end{beamercolorbox}%
  \begin{beamercolorbox}[wd=.333333\paperwidth,ht=2.25ex,dp=1ex,right]{date in head/foot}%
    \usebeamerfont{date in head/foot}\insertshortdate{}\hspace*{2em}
    \insertframenumber{}%
%     / \inserttotalframenumber
    \hspace*{2ex}
  \end{beamercolorbox}}%
  \vskip0pt%
}
\makeatother

% Show the TOC at the beginning
\AtBeginSection[]{
  \addtocounter{framenumber}{-1}
  \begin{frame}[plain]
      \frametitle{Contents}
      \tableofcontents[currentsection,subsectionstyle=shaded/shaded/hide]
  \end{frame}
}

% Show TOC at beginning of each subsection
\AtBeginSubsection[]{
  \addtocounter{framenumber}{-1}
  \begin{frame}[plain]
      \frametitle{Contents}
      \tableofcontents[currentsection, currentsubsection, subsectionstyle=show/shaded/hide]
  \end{frame}
}


\colorlet{beamer@blendedblue}{blue!70} % change color theme

\usepackage[style=numeric-comp,sorting=none,backend=biber]{biblatex}%<- specify style
\addbibresource{references.bib}%<- specify bib file

\usepackage[inkscapearea=page]{svg}
\usepackage{adjustbox}


% For appendix
\newcommand{\backupbegin}{
   \newcounter{framenumberappendix}
   \setcounter{framenumberappendix}{\value{framenumber}}
}
\newcommand{\backupend}{
   \addtocounter{framenumberappendix}{-\value{framenumber}}
   \addtocounter{framenumber}{\value{framenumberappendix}}
}

\setbeamertemplate{bibliography item}{\insertbiblabel} % improved references



% Other preamble stuff:
\usepackage{preamble}

%%% Uncomment for another color palette
% \definecolor{Logo1}{rgb}{0.0, 0, 0.7}
% \definecolor{Logo2}{rgb}{2.55, 2.55, 2.55}

% \setbeamercolor*{palette primary}{bg=Logo1, fg=white}
% \setbeamercolor*{palette secondary}{bg=Logo2, fg=white}
% \setbeamercolor*{palette tertiary}{bg=white, fg=Logo1}
% \setbeamercolor*{palette quaternary}{bg=white,fg=white}
% \setbeamercolor{structure}{fg=Logo1} % itemize, enumerate, etc
% \setbeamercolor{section in toc}{fg=Logo1} % TOC sections

% For figures
\usepackage{import}
\usepackage{xifthen}
\usepackage{pdfpages}
\usepackage{transparent}
\usepackage{mdframed}
\usepackage{subcaption}

\setbeamertemplate{caption}[numbered]

\usepackage{multirow}


% --- Inkscape figures:
\newcommand{\incfig}[2][0.75\textwidth]{%
    \def\svgwidth{\columnwidth}
    \resizebox{#1}{!}{\import{Inkscape/}{#2.pdf_tex}}
}

% --- Height of frame
\newlength{\myheight}
\setlength{\myheight}{7cm}

\newlength\myheightfigureintext
\newlength\mydepthfigureintext
\settototalheight\myheightfigureintext{Xygp}
\settodepth\mydepthfigureintext{Xygp}
\setlength\fboxsep{0pt}

\usepackage{tikz}
\usepackage[absolute,overlay]{textpos} % for precise positioning

\usepackage{ifthen}


%------------------------------------------------------------
%This block of code defines the information to appear in the
%Title page
\title[Neural priors] %optional
{Incorporating neutron star physics into gravitational wave inference with neural priors}

\author[Thibeau Wouters]{\textbf{Thibeau Wouters}, Peter T. H. Pang, Tim Dietrich, Chris Van Den Broeck \\ \vspace{2mm} \href{mailto:t.r.i.wouters@uu.nl}{t.r.i.wouters@uu.nl} \newline \href{https://arxiv.org/abs/2511.22987}{\texttt{arXiv:2511.22987}} \newline \github \quad \linkedin \quad \myemail}

\date{GWfreeride 2026}

%End of title page configuration block
%------------------------------------------------------------



%------------------------------------------------------------
%The next block of commands puts the table of contents at the
%beginning of each section and highlights the current section:

%------------------------------------------------------------


\begin{document}

{
\usebackgroundtemplate{\transparent{0.5}{\includegraphics[width=\paperwidth,height=\paperheight]{Figures/tintin_BNS_2.png}}}

\begin{frame}[plain, noframenumbering]

  \begin{tikzpicture}[remember picture,overlay]
    \node[fill=customblue, fill opacity=0.75, text opacity=1, rounded corners=10pt, inner sep=15pt] at (current page.center) {
      \begin{minipage}{0.8\textwidth}
        \centering
        \textbf{Incorporating neutron star physics into gravitational wave inference with neural priors}\\[1.5ex]
        \small Thibeau Wouters, Peter T. H. Pang, Tim Dietrich, Chris Van Den Broeck \normalsize \\[0.5ex]
        {\hypersetup{hidelinks}
         \href{https://arxiv.org/abs/2511.22987}{\texttt{arXiv:2511.22987}}
        } \\[0.5ex]
        \github \quad \linkedin \quad \myemail
      \end{minipage}
    };
  \end{tikzpicture}

  \vspace{7cm}

  \begin{columns}
  \column{0.35\textwidth}
  \begin{figure}
    \centering
    \vspace{1.5mm}
    \includegraphics[width=0.75\linewidth]{Figures/utrecht-university.png}
  \end{figure}
  \column{0.35\textwidth}
  \begin{figure}
    \centering
    \includegraphics[width=0.75\linewidth]{Figures/Nikhef_logo-transparent.png}
  \end{figure}
\end{columns}

  \end{frame}
}

\begin{frame}{Motivation}
  \def\x{3mm}
  \def\y{1mm}

  \begin{itemize}
    \item Posterior $\propto$ likelihood $\times$ \red{prior}
    
    \vspace{\x}
    
    \item \red{Prior}: usually agnostic, but can encode valuable information!
    \begin{itemize}
      \vspace{\y}
      \item Theoretical understanding

      \vspace{\y}

      \item Observations outside of GW
    \end{itemize}

    \pause
    \vspace{\x}

    \item \textbf{Neutron stars} -- key GW observables?
    \begin{itemize}
      \vspace{\y}
      \item Masses $m_1, m_2$

      \vspace{\y}

      \item Tidal deformabilities $\Lambda_1, \Lambda_2$ $\leftarrow$ equation of state (EOS)
    \end{itemize}

    \vspace{\x}
    \incfig[0.80\textwidth]{tidal}

    % Example: neutron stars (NSs)
    % \begin{itemize}
    %   \item Equation of state (EOS): pressure vs density inside NS
    %   \item Tidal deformability $\Lambda$ depends on EOS
    %   \item Masses:
    % \end{itemize}

    % \vspace{\x}

    % \item Emulate data-driven priors with normalizing flows~\cite{Alsing:2021wef}

    % \vspace{\x}
    
    % \item Our work (\texttt{arXiv:2511.22987}): apply this to neutron star physics in GW
  \end{itemize}
\end{frame}

\begin{frame}{Key idea}
  \def\x{3mm}

  \begin{figure}
    \includegraphics[width=0.80\linewidth]{Figures/Figure1_presentation.pdf}
  \end{figure}
\end{frame}


\begin{frame}{Example prior}
  \def\x{3mm}

  \begin{figure}
    \includegraphics[width=0.60\linewidth]{Figures/bns_gaussian_all_eos_chirp_tilde.pdf}
  \end{figure}
\end{frame}

\begin{frame}{GW170817 posteriors with EOS informed priors}
  \def\x{3mm}

  \begin{figure}
    \includegraphics[width=0.65\linewidth]{Figures/GW170817_corner_gaussian_bns.pdf}
  \end{figure}
\end{frame}

\begin{frame}{Closing thoughts}
  \def\y{4mm}

  \begin{itemize}
    \item \jaxtwo{\textbf{Neural priors}}: encode non-trivial prior information (including uncertainties)

    \vspace{\y}
    
    \item Implemented in \textsc{bilby}, but works with any stochastic sampler
    \begin{itemize}
      \item Appealing for GPU-based samplers (\textsc{flowMC}, \textsc{blackjax})
    \end{itemize}

    \vspace{\y}
    
    \item Bayesian source classification:
    \begin{itemize}
      \item GW170817: BNS, soft EOS
      \item GW230529: NSBH favored over BNS
    \end{itemize}
    
    \vspace{\y}
    
    \item ML for `informed' sampling (cf. Michael Williams's talk)
    
    \vspace{\y}
    
    \item Code is open source
    \begin{itemize}
      \item \scriptsize \textcolor{black}{\faGithub} \url{ThibeauWouters/neural-priors} \normalsize
      \item \scriptsize \textcolor{black}{\faGithub} \url{ThibeauWouters/bilby/tree/neural_prior_bilby_pipe} \normalsize
    \end{itemize}
    
  \end{itemize}
\end{frame}

{
\usebackgroundtemplate{\transparent{0.5}{\includegraphics[width=\paperwidth,height=\paperheight]{Figures/tintin_BNS_2.png}}}

\begin{frame}[plain, noframenumbering]

  \begin{tikzpicture}[remember picture,overlay]
    \node[fill=customblue, fill opacity=0.75, text opacity=1, rounded corners=10pt, inner sep=15pt] at ([yshift=2cm]current page.center) {
      \begin{minipage}{0.8\textwidth}
        \centering
        \textbf{Thanks for listening!}
      \end{minipage}
    };
  \end{tikzpicture}

  \end{frame}
}

\end{document}
