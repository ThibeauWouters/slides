%%% ======= Beamer ======
\documentclass[usenames,dvipsnames,t]{beamer}
% \documentclass[usenames,dvipsnames, handout]{beamer}
\beamertemplatenavigationsymbolsempty % remove toolbar at the bottom of slides
\usepackage{appendixnumberbeamer} % for appendix
\usetheme{Madrid}
\usecolortheme{default}
\useinnertheme{circles}

\usepackage{fontawesome}
\usepackage{colortbl}  % For \cellcolor in tables

% Define commands for social media icons with links
\newcommand{\linkedin}{\href{https://www.linkedin.com/in/ThibeauWouters}{\textcolor{black}{\faLinkedin}}}
\newcommand{\github}{\href{https://github.com/ThibeauWouters}{\textcolor{black}{\faGithub}}}
\newcommand{\myemail}{\href{mailto:t.r.i.wouters@uu.nl}{\textcolor{black}{\faEnvelope}}}

\newcommand{\ghlink}[1]{\href{https://github.com/#1}{\textcolor{black}{\faGithub}}}

\definecolor{customblue}{HTML}{7db8dc}
\newcommand{\thetaeos}{\boldsymbol{\theta}_{\rm{EOS}}}
\newcommand{\boldtheta}{\boldsymbol{\theta}}

% Color definitions for Jeffrey's scale interpretation
% Taken from part of "rocket" color scheme of Seaborn
\definecolor{jeffreysred1}{HTML}{f6cdb0}  % barely worth mentioning
\definecolor{jeffreysred2}{HTML}{fc9074}  % substantial
\definecolor{jeffreysred3}{HTML}{f4744c}  % strong
\definecolor{jeffreysred4}{HTML}{ef5b43}  % very strong
\definecolor{jeffreysred5}{HTML}{f52a44}  % decisive



\setbeamercolor{author in head/foot}{bg=blue!10, fg=blue}
\setbeamercolor{title in head/foot}{bg=blue!10, fg=blue}
\setbeamercolor{date in head/foot}{bg=blue!10, fg=blue}

\makeatletter
\setbeamertemplate{footline}{
  \leavevmode%
  \hbox{%
  \begin{beamercolorbox}[wd=.333333\paperwidth,ht=2.25ex,dp=1ex,center]{author in head/foot}%
    \usebeamerfont{author in head/foot}\insertshortauthor\expandafter\ifblank\expandafter{\beamer@shortinstitute}{}{~~(\insertshortinstitute)}
  \end{beamercolorbox}%
  \begin{beamercolorbox}[wd=.333333\paperwidth,ht=2.25ex,dp=1ex,center]{title in head/foot}%
    \usebeamerfont{title in head/foot}\insertshorttitle
  \end{beamercolorbox}%
  \begin{beamercolorbox}[wd=.333333\paperwidth,ht=2.25ex,dp=1ex,right]{date in head/foot}%
    \usebeamerfont{date in head/foot}\insertshortdate{}\hspace*{2em}
    \insertframenumber{}%
%     / \inserttotalframenumber
    \hspace*{2ex}
  \end{beamercolorbox}}%
  \vskip0pt%
}
\makeatother

% Show the TOC at the beginning
\AtBeginSection[]{
  \addtocounter{framenumber}{-1}
  \begin{frame}[plain]
      \frametitle{Contents}
      \tableofcontents[currentsection,subsectionstyle=shaded/shaded/hide]
  \end{frame}
}

% Show TOC at beginning of each subsection
\AtBeginSubsection[]{
  \addtocounter{framenumber}{-1}
  \begin{frame}[plain]
      \frametitle{Contents}
      \tableofcontents[currentsection, currentsubsection, subsectionstyle=show/shaded/hide]
  \end{frame}
}


\colorlet{beamer@blendedblue}{blue!70} % change color theme

\usepackage[style=numeric-comp,sorting=none,backend=biber]{biblatex}%<- specify style
\addbibresource{references.bib}%<- specify bib file

\usepackage[inkscapearea=page]{svg}
\usepackage{adjustbox}


% For appendix
\newcommand{\backupbegin}{
   \newcounter{framenumberappendix}
   \setcounter{framenumberappendix}{\value{framenumber}}
}
\newcommand{\backupend}{
   \addtocounter{framenumberappendix}{-\value{framenumber}}
   \addtocounter{framenumber}{\value{framenumberappendix}}
}

\setbeamertemplate{bibliography item}{\insertbiblabel} % improved references



% Other preamble stuff:
\usepackage{preamble}

%%% Uncomment for another color palette
% \definecolor{Logo1}{rgb}{0.0, 0, 0.7}
% \definecolor{Logo2}{rgb}{2.55, 2.55, 2.55}

% \setbeamercolor*{palette primary}{bg=Logo1, fg=white}
% \setbeamercolor*{palette secondary}{bg=Logo2, fg=white}
% \setbeamercolor*{palette tertiary}{bg=white, fg=Logo1}
% \setbeamercolor*{palette quaternary}{bg=white,fg=white}
% \setbeamercolor{structure}{fg=Logo1} % itemize, enumerate, etc
% \setbeamercolor{section in toc}{fg=Logo1} % TOC sections

% For figures
\usepackage{import}
\usepackage{xifthen}
\usepackage{pdfpages}
\usepackage{transparent}
\usepackage{mdframed}
\usepackage{subcaption}

\setbeamertemplate{caption}[numbered]

\usepackage{multirow}


% --- Inkscape figures:
\newcommand{\incfig}[2][0.75\textwidth]{%
    \def\svgwidth{\columnwidth}
    \resizebox{#1}{!}{\import{Inkscape/}{#2.pdf_tex}}
}

% --- Height of frame
\newlength{\myheight}
\setlength{\myheight}{7cm}

\newlength\myheightfigureintext
\newlength\mydepthfigureintext
\settototalheight\myheightfigureintext{Xygp}
\settodepth\mydepthfigureintext{Xygp}
\setlength\fboxsep{0pt}

\usepackage{tikz}
\usepackage[absolute,overlay]{textpos} % for precise positioning

\usepackage{ifthen}


%------------------------------------------------------------
%This block of code defines the information to appear in the
%Title page
\title[Neural priors] %optional
{Incorporating neutron star physics into gravitational wave inference with neural priors}

\author[Thibeau Wouters]{\textbf{Thibeau Wouters}, Peter T. H. Pang, Tim Dietrich, Chris Van Den Broeck \\ \vspace{2mm} \href{mailto:t.r.i.wouters@uu.nl}{t.r.i.wouters@uu.nl} \newline \href{https://arxiv.org/abs/2511.22987}{\texttt{arXiv:2511.22987}} \newline \github \quad \linkedin \quad \myemail}

\date{GRANDMA call 15/01/2026}

%End of title page configuration block
%------------------------------------------------------------



%------------------------------------------------------------
%The next block of commands puts the table of contents at the
%beginning of each section and highlights the current section:

%------------------------------------------------------------


\begin{document}

{
\usebackgroundtemplate{\transparent{0.5}{\includegraphics[width=\paperwidth,height=\paperheight]{Figures/tintin_BNS_2.png}}}

\begin{frame}[plain, noframenumbering]

  \begin{tikzpicture}[remember picture,overlay]
    \node[fill=customblue, fill opacity=0.75, text opacity=1, rounded corners=10pt, inner sep=15pt] at (current page.center) {
      \begin{minipage}{0.8\textwidth}
        \centering
        \textbf{Incorporating neutron star physics into gravitational wave inference with neural priors}\\[1.5ex]
        \small Thibeau Wouters, Peter T. H. Pang, Tim Dietrich, Chris Van Den Broeck \normalsize \\[0.5ex]
        {\hypersetup{hidelinks}
         \href{https://arxiv.org/abs/2511.22987}{\texttt{arXiv:2511.22987}}
        } \\[0.5ex]
        \github \quad \linkedin \quad \myemail
      \end{minipage}
    };
  \end{tikzpicture}

  \vspace{7cm}

  \begin{columns}
  \column{0.35\textwidth}
  \begin{figure}
    \centering
    \vspace{1.5mm}
    \includegraphics[width=0.75\linewidth]{Figures/utrecht-university.png}
  \end{figure}
  \column{0.35\textwidth}
  \begin{figure}
    \centering
    \includegraphics[width=0.75\linewidth]{Figures/Nikhef_logo-transparent.png}
  \end{figure}
\end{columns}

  \end{frame}
}


\begin{frame}
\frametitle{Table of Contents}
\tableofcontents[hideallsubsections]
\end{frame}

\section{Introduction}

\begin{frame}{Warm-up example: Motivation}
  \def\x{5mm}
  \def\y{3mm}

  \textbf{``What is the probability that the Sun will rise tomorrow?''}
  \pause
  \vspace{\x}

  Your \red{prior belief} (before seeing tomorrow's data) can be:

  \begin{itemize}
    \item Agnostic: $p(\rm{}\rm{rise}) = 0.5$, $p(\rm{}\rm{not~rise}) = 0.5$
    
    \vspace{\y}
    \pause
    
    \item Informed: $p(\rm{}\rm{rise}) \approx 1$, $p(\rm{}\rm{not~rise}) \approx 0$
    \begin{itemize}
      \item Theory: celestial mechanics
      \item Past observations: every day so far
    \end{itemize}
  \end{itemize}

  \vspace{\x}
  \pause

  Can we do something similar for gravitational wave (GW) signals involving neutron stars (NSs)?
\end{frame}

\begin{frame}{Bayesian inference}
  \def\x{3mm}
  \def\y{1mm}

  \textbf{How does GW parameter estimation work?}
  
  \vspace{\x}
  
  Bayesian inference:

  \begin{equation*}
    \mathcal{P}(\theta_{\rm{GW}} | d ) = \frac{\mathcal{L}(d | \theta_{\rm{GW}}) \red{\pi(\theta_{\rm{GW}})}}{\mathcal{Z}}
  \end{equation*}

  \vspace{\x}

  \begin{itemize}
    \vspace{\y}
    \item Posterior $\mathcal{P}(\theta_{\rm{GW}} | d )$: probability of parameters $\theta_{\rm{GW}}$ given data $d$

    \vspace{\y}

    \item Likelihood $\mathcal{L}(d | \theta_{\rm{GW}})$: probability of data $d$ given parameters $\theta_{\rm{GW}}$ and a waveform model

    \vspace{\y}

    \item \red{Prior} \red{$\pi(\theta_{\rm{GW}})$}: initial belief about parameters $\theta_{\rm{GW}}$ before seeing data $d$
  \end{itemize}

  \vspace{\x}
  \pause

  Our final parameter estimates are a ``mixture'' of prior beliefs and information from the data.

\end{frame}

\begin{frame}{Motivation}
  \def\x{3mm}
  \def\y{1mm}

  \begin{itemize}
    \item Bayesian inference crucially depends on \red{priors}:
    \begin{equation*}
      \mathcal{P}(\theta_{\rm{GW}} | d ) \propto \frac{\mathcal{L}(d | \theta_{\rm{GW}}) \red{\pi(\theta_{\rm{GW}})}}{\mathcal{Z}}
    \end{equation*}

    \vspace{\x}
    
    \item By default, LVK uses \red{agnostic priors}, but what if we \textit{do} have non-trivial prior information? (Theory and observations)
    
    \vspace{\x}

    \item Case study in this work: neutron stars (NSs) and information from
    \begin{itemize}
      \item Population models
      \item Equation of state (EOS) constraints
    \end{itemize}
  \end{itemize}

  \vspace{\x}

  \begin{tcolorbox}[colback=blue!5!white,colframe=blue!75!black,title=Neural priors]
    Flexible way to encode NS physics into GW inference
  \end{tcolorbox}
\end{frame}

\begin{frame}{Tidal deformability}
  \def\x{4mm}

  \begin{itemize}
    \item Neutron stars are tidally deformed in a binary
    
    \vspace{\x}
    
    \item Quantified by the tidal deformability $\Lambda$

    \vspace{\x}

    \item Depends on the equation of state: $\Lambda = \Lambda(m, \rm{EOS})$

    \vspace{\x}

    \item Imprint in the GW phase: $\tilde{\Lambda}(m_i, \Lambda_i)$ %  $\rightarrow$ measurable $\rightarrow$ EOS constraints
  \end{itemize}

  \begin{figure}
    \centering
    \incfig[1.0\textwidth]{tidal}
  \end{figure}
\end{frame}


\begin{frame}{Key idea}
  \def\x{3mm}

  Train normalizing flow (NF) on samples informed by populations and EOS
  $\rightarrow$ \jaxtwo{\textbf{neural prior}}

  \begin{figure}
    \centering
    \includegraphics[width=0.55\linewidth]{Figures/Figure1.pdf}
  \end{figure}
\end{frame}


\section{Methods}

\begin{frame}{NS population models}
  \def\x{3.5mm}

  Three fiducial population models for NS masses:

  \vspace{\x}

  \begin{enumerate}
    \item \textbf{Uniform}~\cite{LIGOScientific:2021qlt, Landry:2021hvl, Golomb:2024lds}:
    \begin{itemize}
      \item Only use EOS constraints for maximum mass ($\MTOV$)
      \item NS mass $\sim U[1\,\Msun, M_{\rm{TOV}}]$
    \end{itemize}

    \vspace{\x}

    \item \textbf{Gaussian}~\cite{Ozel:2016oaf}:
    \begin{itemize}
      \item NS mass $\sim \mathcal{N}(1.33\,\Msun, (0.09\,\Msun)^2)$
    \end{itemize}

    \vspace{\x}

    \item \textbf{Double Gaussian}~\cite{Alsing:2017bbc, Shao:2020bzt}:
    \begin{itemize}
      \item Weighted mixture of two Gaussians
      \item $0.65 \times \mathcal{N}(1.34\,\Msun, (0.07\,\Msun)^2) + 0.35 \times \mathcal{N}(1.80\,\Msun, (0.21\,\Msun)^2)$
    \end{itemize}
  \end{enumerate}

  \vspace{\x}

  For neutron star-black hole (NSBH) systems:
  \begin{itemize}
    \item BH mass $m_1^{\rm{src}}$: from $[\MTOV, 5\,\Msun]$
    \item NS mass $m_2^{\rm{src}}$: above models
  \end{itemize}
\end{frame}

\begin{frame}{Equation of state constraints}

  \def\x{2mm}
  \def\y{2mm}

  \begin{itemize}
    \item We use three equation of state constraints~\cite{Koehn:2024set}:

    \begin{enumerate}
      \vspace{\y}
      \item \textbf{Heavy pulsars}: must support $2\,M_{\odot}$ NSs

      \vspace{\y}

      \item \textbf{Chiral EFT} ($\chi_{\rm{EFT}}$): nuclear theory predictions (softer EOS)

      \vspace{\y}

      \item \textbf{NICER}: mass-radius observations of NSs (stiffer EOS)
    \end{enumerate}

    \vspace{\x}

    \item Posterior EOS samples obtained with \textsc{jester}~\cite{Wouters:2025zju}~\ghlink{nuclear-multimessenger-astronomy/jester}
  \end{itemize}

  \vspace{-1mm}

  \centering
  \incfig[0.95\textwidth]{R14_table}
\end{frame}

\begin{frame}{Normalizing flows}
  \def\x{3mm}

  Normalizing flows~\cite{Kobyzev:2019ydm, Papamakarios:2019fms}

  \begin{itemize}
    \item Neural density estimators: trained on samples (predictions)
    \item Generate samples and evaluate density
    \item Can be used as priors in nested sampling!~\cite{Alsing:2021wef}
  \end{itemize}

  % \vspace{\x}

  \centering
  \incfig[0.85\textwidth]{NF}
\end{frame}

\begin{frame}{Construction of neural priors}
  \def\x{5mm}
  \def\y{2mm}

  Steps to generate training data:
  \begin{enumerate}
    \vspace{\y}
    
    \item Draw EOS posterior curve: determines $\MTOV$, $\Lambda(m)$
    
    \vspace{\y}
    
    \item Draw masses from population model

    \vspace{\y}
    
    \item Compute $\Lambda_i = \Lambda(m_i)$ for NSs (NSBH: $\Lambda_1 = 0$)
  \end{enumerate}

  \vspace{\x}
  \pause

  Training:
  \begin{itemize}
    \vspace{\y}

    \item Repeat procedure $N_{\rm{training}}$ times to get training set
    
    \vspace{\y}

    \item Train normalizing flow to approximate density
  \end{itemize}
\end{frame}

\begin{frame}{All neural priors}
  \vspace{-1mm}

  \begin{figure}
    \centering
    \includegraphics[width=0.99\linewidth]{Figures/bns_nsbh_all_populations_chirp_tilde.pdf}
  \end{figure}
\end{frame}

\section{Results}

\begin{frame}{Setup}

  \def\x{3mm}

  Analyze GW170817, GW190425, GW230529 with:
  \begin{itemize}
    \item $4096$ live points, multibanding likelihood
    
    \item \texttt{IMRPhenomXP\_NRTidalv3}
    
    \item Neural priors for $m_i$, $\Lambda_i$ (standard priors for other parameters)
  \end{itemize}

  \vspace{\x}

  Two contributions:
  \begin{enumerate}
    \item Narrower constraints with neural priors
    \item Model selection with Bayes factors
  \end{enumerate}

  \vspace{\x}

  \begin{table}
    \centering
    \caption*{\textbf{Jeffreys' scale for Bayes factors} ($\log_{10}$ scale)}
    \small
    \begin{tabular}{ccc}
      \hline
      $\log_{10}(\mathcal{B}_1^2)$ & Interpretation & Color \\
      \hline
      $[0,\, \tfrac{1}{2}]$ & Barely worth mentioning & \cellcolor{jeffreysred1} \\
      $[\tfrac{1}{2},\, 1]$ & Substantial & \cellcolor{jeffreysred2} \\
      $[1,\, \tfrac{3}{2}]$ & Strong & \cellcolor{jeffreysred3} \\
      $[\tfrac{3}{2},\, 2]$ & Very strong & \cellcolor{jeffreysred4} \\
      $> 2$ & Decisive & \cellcolor{jeffreysred5} \\
      \hline
    \end{tabular}
  \end{table}

\end{frame}

\subsection{GW170817}

\begin{frame}{GW170817: Source classification}

  Showing $\log_{10}$ Bayes factors relative to model with highest evidence

  \vspace{3mm}

  \begin{table}
    \centering
    \scriptsize
    \begin{tabular}{|l|l|l|c|}
    \hline
    \textbf{Source} & \textbf{Population} & \textbf{EOS} & \textbf{GW170817} \\
    \hline\hline
    \multirow{9}{*}{BNS} & \multirow{3}{*}{Uniform} & PSRs & \cellcolor{jeffreysred4}$-1.83$ \\
    \cline{3-4}
     &  & PSRs+$\chi_{\rm{EFT}}$ & \cellcolor{jeffreysred2}$-0.80$ \\
    \cline{3-4}
     &  & PSRs+NICER & \cellcolor{jeffreysred4}$-1.58$ \\
    \cline{2-4}
     & \multirow{3}{*}{Gaussian} & PSRs & \cellcolor{jeffreysred2}$-0.68$ \\
    \cline{3-4}
     &  & PSRs+$\chi_{\rm{EFT}}$ & \textbf{ref.} \\
    \cline{3-4}
     &  & PSRs+NICER & \cellcolor{jeffreysred2}$-0.76$ \\
    \cline{2-4}
     & \multirow{3}{*}{Double Gaussian} & PSRs & \cellcolor{jeffreysred3}$-1.36$ \\
    \cline{3-4}
     &  & PSRs+$\chi_{\rm{EFT}}$ & \cellcolor{jeffreysred2}$-0.59$ \\
    \cline{3-4}
     &  & PSRs+NICER & \cellcolor{jeffreysred2}$-0.92$ \\
    \hline\hline
    \multirow{9}{*}{NSBH} & \multirow{3}{*}{Uniform} & PSRs & \cellcolor{jeffreysred5}$-224.65$ \\
    \cline{3-4}
     &  & PSRs+$\chi_{\rm{EFT}}$ & \cellcolor{jeffreysred5}$-224.66$ \\
    \cline{3-4}
     &  & PSRs+NICER & \cellcolor{jeffreysred5}$-224.66$ \\
    \cline{2-4}
     & \multirow{3}{*}{Gaussian} & PSRs & \cellcolor{jeffreysred5}$-224.67$ \\
    \cline{3-4}
     &  & PSRs+$\chi_{\rm{EFT}}$ & \cellcolor{jeffreysred5}$-224.66$ \\
    \cline{3-4}
     &  & PSRs+NICER & \cellcolor{jeffreysred5}$-224.66$ \\
    \cline{2-4}
     & \multirow{3}{*}{Double Gaussian} & PSRs & \cellcolor{jeffreysred5}$-224.67$ \\
    \cline{3-4}
     &  & PSRs+$\chi_{\rm{EFT}}$ & \cellcolor{jeffreysred5}$-224.68$ \\
    \cline{3-4}
     &  & PSRs+NICER & \cellcolor{jeffreysred5}$-224.67$ \\
    \hline
    \end{tabular}
  \end{table}
\end{frame}

\begin{frame}{GW170817: Parameter constraints (Gaussian)}

  \vspace{-3mm}
  \begin{figure}
    \centering
    \includegraphics[width=0.69\linewidth]{Figures/GW170817_corner_gaussian_bns.pdf}
  \end{figure}
\end{frame}

\begin{frame}{GW170817: Discussion}
  \def\x{5mm}

  \textbf{Source classification:}
  \begin{itemize}
    \item Decisive evidence for BNS over NSBH
    \item Prefer Gaussian population model
    \item Slight preference for softer EOS (PSRs+$\chi_{\rm{EFT}}$)
  \end{itemize}

  \vspace{\x}

  \textbf{Parameter constraints:}
  \begin{itemize}
    \item More equal mass ratio: $q \geq 0.9$
    \item Constrained tidal deformability $\tilde{\Lambda}$
    \item Higher luminosity distance compared to agnostic prior
    \item Matches multimessenger analyses of GW170817~\cite{Radice:2018ozg, Coughlin:2018fis, LIGOScientific:2018hze, Pang:2022rzc, Breschi:2024qlc}
  \end{itemize}
\end{frame}

\subsection{GW190425}

\begin{frame}{GW190425: Source classification}

  Showing $\log_{10}$ Bayes factors relative to best model

  \vspace{3mm}

  \begin{table}
    \centering
    \scriptsize
    \begin{tabular}{|l|l|l|c|}
    \hline
    \textbf{Source} & \textbf{Population} & \textbf{EOS} & \textbf{GW190425} \\
    \hline\hline
    \multirow{9}{*}{BNS} & \multirow{3}{*}{Uniform} & PSRs & \cellcolor{jeffreysred1}$-0.07$ \\
    \cline{3-4}
     &  & PSRs+$\chi_{\rm{EFT}}$ & \cellcolor{jeffreysred1}$-0.11$ \\
    \cline{3-4}
     &  & PSRs+NICER & \textbf{ref.} \\
    \cline{2-4}
     & \multirow{3}{*}{Gaussian} & PSRs & \cellcolor{jeffreysred5}$-6.89$ \\
    \cline{3-4}
     &  & PSRs+$\chi_{\rm{EFT}}$ & \cellcolor{jeffreysred5}$-8.47$ \\
    \cline{3-4}
     &  & PSRs+NICER & \cellcolor{jeffreysred5}$-5.45$ \\
    \cline{2-4}
     & \multirow{3}{*}{Double Gaussian} & PSRs & \cellcolor{jeffreysred2}$-0.55$ \\
    \cline{3-4}
     &  & PSRs+$\chi_{\rm{EFT}}$ & \cellcolor{jeffreysred2}$-0.79$ \\
    \cline{3-4}
     &  & PSRs+NICER & \cellcolor{jeffreysred2}$-0.57$ \\
    \hline\hline
    \multirow{9}{*}{NSBH} & \multirow{3}{*}{Uniform} & PSRs & \cellcolor{jeffreysred4}$-1.52$ \\
    \cline{3-4}
     &  & PSRs+$\chi_{\rm{EFT}}$ & \cellcolor{jeffreysred3}$-1.35$ \\
    \cline{3-4}
     &  & PSRs+NICER & \cellcolor{jeffreysred4}$-1.63$ \\
    \cline{2-4}
     & \multirow{3}{*}{Gaussian} & PSRs & \cellcolor{jeffreysred2}$-0.82$ \\
    \cline{3-4}
     &  & PSRs+$\chi_{\rm{EFT}}$ & \cellcolor{jeffreysred3}$-1.11$ \\
    \cline{3-4}
     &  & PSRs+NICER & \cellcolor{jeffreysred3}$-1.43$ \\
    \cline{2-4}
     & \multirow{3}{*}{Double Gaussian} & PSRs & \cellcolor{jeffreysred5}$-4.11$ \\
    \cline{3-4}
     &  & PSRs+$\chi_{\rm{EFT}}$ & \cellcolor{jeffreysred5}$-3.83$ \\
    \cline{3-4}
     &  & PSRs+NICER & \cellcolor{jeffreysred5}$-24.31$ \\
    \hline
    \end{tabular}
  \end{table}
\end{frame}

\begin{frame}{GW190425: Parameter constraints (Uniform)}

  \vspace{-3mm}

  \begin{figure}
    \centering
    \includegraphics[width=0.69\linewidth]{Figures/GW190425_corner_uniform_bns.pdf}
  \end{figure}
\end{frame}

\begin{frame}{GW190425: Discussion}
  \def\x{5mm}

  \textbf{Source classification:}
  \begin{itemize}
    \item Prefer BNS over NSBH (but less conclusive than GW170817)
    \item Most consistent with uniform population
    \item Masses are outliers compared to galactic binaries
  \end{itemize}

  \vspace{\x}

  \textbf{Parameter constraints:}
  \begin{itemize}
    \item Less equal masses: $q \leq 0.9$
    \item $\tilde{\Lambda} \approx 200$ (more prior-dominated due to lower SNR)
    \item Higher luminosity distance: $182^{+41}_{-49}$ Mpc vs. $157^{+64}_{-65}$ Mpc ($90\%$ credibility)
  \end{itemize}
\end{frame}

\subsection{GW230529}

\begin{frame}{GW230529: Source classification}

  Showing $\log_{10}$ Bayes factors relative to best model

  \vspace{3mm}

  \begin{table}
    \centering
    \scriptsize
    \begin{tabular}{|l|l|l|c|}
    \hline
    \textbf{Source} & \textbf{Population} & \textbf{EOS} & \textbf{GW230529} \\
    \hline\hline
    \multirow{9}{*}{BNS} & \multirow{3}{*}{Uniform} & PSRs & \cellcolor{jeffreysred5}$-13.14$ \\
    \cline{3-4}
     &  & PSRs+$\chi_{\rm{EFT}}$ & \cellcolor{jeffreysred5}$-13.12$ \\
    \cline{3-4}
     &  & PSRs+NICER & \cellcolor{jeffreysred5}$-12.92$ \\
    \cline{2-4}
     & \multirow{3}{*}{Gaussian} & PSRs & \cellcolor{jeffreysred5}$-18.82$ \\
    \cline{3-4}
     &  & PSRs+$\chi_{\rm{EFT}}$ & \cellcolor{jeffreysred5}$-18.83$ \\
    \cline{3-4}
     &  & PSRs+NICER & \cellcolor{jeffreysred5}$-18.81$ \\
    \cline{2-4}
     & \multirow{3}{*}{Double Gaussian} & PSRs & \cellcolor{jeffreysred5}$-13.75$ \\
    \cline{3-4}
     &  & PSRs+$\chi_{\rm{EFT}}$ & \cellcolor{jeffreysred5}$-13.77$ \\
    \cline{3-4}
     &  & PSRs+NICER & \cellcolor{jeffreysred5}$-13.71$ \\
    \hline\hline
    \multirow{9}{*}{NSBH} & \multirow{3}{*}{Uniform} & PSRs & \cellcolor{jeffreysred1}$-0.08$ \\
    \cline{3-4}
     &  & PSRs+$\chi_{\rm{EFT}}$ & \cellcolor{jeffreysred1}$-0.02$ \\
    \cline{3-4}
     &  & PSRs+NICER & \cellcolor{jeffreysred1}$-0.25$ \\
    \cline{2-4}
     & \multirow{3}{*}{Gaussian} & PSRs & \cellcolor{jeffreysred1}$-0.05$ \\
    \cline{3-4}
     &  & PSRs+$\chi_{\rm{EFT}}$ & \cellcolor{jeffreysred1}$-0.20$ \\
    \cline{3-4}
     &  & PSRs+NICER & \textbf{ref.} \\
    \cline{2-4}
     & \multirow{3}{*}{Double Gaussian} & PSRs & \cellcolor{jeffreysred1}$-0.14$ \\
    \cline{3-4}
     &  & PSRs+$\chi_{\rm{EFT}}$ & \cellcolor{jeffreysred1}$-0.13$ \\
    \cline{3-4}
     &  & PSRs+NICER & \cellcolor{jeffreysred1}$-0.05$ \\
    \hline
    \end{tabular}
  \end{table}
\end{frame}

\begin{frame}{GW230529: Parameter constraints (Gaussian)}

  \vspace{-3mm}

  \begin{figure}
    \centering
    \includegraphics[width=0.68\linewidth]{Figures/GW230529_corner_gaussian_nsbh.pdf}
  \end{figure}
\end{frame}

\begin{frame}{GW230529: Discussion}
  \def\x{5mm}

  \textbf{Source classification:}
  \begin{itemize}
    \item Decisive evidence for NSBH over BNS (agrees with LVK~\cite{LIGOScientific:2024elc})
    \item No evidence between hypotheses (low SNR)
  \end{itemize}

  \vspace{\x}

  \textbf{Parameter constraints:}
  \begin{itemize}
    \item Mass ratio more constrained: $q \leq 0.4$
    \begin{itemize}
      \item As a result, improved spin constraints ($\chi_{1z}$ closer to zero)
    \end{itemize}
    \item Tidal deformability posteriors dominated by priors
    \item Luminosity distance: $235^{+59}_{-58}$ Mpc vs. $201^{+84}_{-97}$ Mpc ($90\%$ credibility)
    \begin{itemize}
      \item Important for EM follow-up studies
    \end{itemize}
    \item Less tidal information: \texttt{NRTidalv3} tapers at ($90\%$ credibility)
    \begin{itemize}
      \item $641_{-158}^{+318}$ Hz (agnostic priors)
      \item $858_{-108}^{+113}$ Hz (neural prior, reference model)
    \end{itemize}
  \end{itemize}
\end{frame}

\section{Conclusion}

\begin{frame}{Conclusion}
  \def\y{3mm}

  \begin{itemize}
    \item Agnostic priors miss including valuable information

    \vspace{\y}
    
    \item \jaxtwo{\textbf{Neural priors}}: flexible way to encode non-trivial prior information
    
    \vspace{\y}
    
    \item Case study: neutron star physics from
    \begin{itemize}
      \item Population models
      \item EOS constraints
    \end{itemize}

    \vspace{\y}

    \item We consistelty recover higher luminosity distances with informed priors compared to the agnostic priors
    \begin{itemize}
      \item Important for EM follow-up
      \item Simulation studies ongoing to quantify this
    \end{itemize}

    \vspace{\y}

    \item Data-driven approach: easy to extend/generalize!
  \end{itemize}
\end{frame}

{
\usebackgroundtemplate{\transparent{0.5}{\includegraphics[width=\paperwidth,height=\paperheight]{Figures/tintin_BNS_2.png}}}

\begin{frame}[plain, noframenumbering]

  \begin{tikzpicture}[remember picture,overlay]
    \node[fill=customblue, fill opacity=0.75, text opacity=1, rounded corners=10pt, inner sep=15pt] at ([yshift=2cm]current page.center) {
      \begin{minipage}{0.8\textwidth}
        \centering
        \textbf{Thanks for listening!}
      \end{minipage}
    };
  \end{tikzpicture}

  \end{frame}
}

\begin{frame}[allowframebreaks]{References}
  \printbibliography
\end{frame}

\appendix

\begin{frame}{Likelihood distributions}
  \begin{columns}[t]
    \column{0.5\textwidth}
    
    \begin{itemize}
      \item Likelihood distributions, obtained from final posterior samples
    \end{itemize}

    \column{0.5\textwidth}

    \vspace{-10mm}

    \begin{figure}
      \centering
      \includegraphics[width=0.90\linewidth]{Figures/combined_log_likelihood_three_events.pdf}
    \end{figure}
  \end{columns}
\end{frame}

\begin{frame}{Information gain}

  \def\x{3mm}

  \begin{columns}[t]
    \column{0.5\textwidth}
    
    \begin{itemize}
      \item ``Prior'' = $\Lambda_i$ computed from EOSs conditioned on the posterior source-frame masses $m_i^{\rm{src}}$

      \vspace{\x}

      \item ``Posterior'' = $\Lambda_i$ from posterior samples

      \vspace{\x}
      
      \item KL divergence between prior and posterior in brackets
    \end{itemize}

    \column{0.5\textwidth}

    \vspace{-10mm}

    \begin{figure}
      \centering
      \includegraphics[width=0.90\linewidth]{Figures/combined_prior_vs_posterior.pdf}
    \end{figure}
  \end{columns}
\end{frame}

\begin{frame}{More posteriors}
  \vspace{-4mm}
  \begin{figure}
    \centering
    \includegraphics[width=0.875\linewidth]{Figures/big_table.jpg}
  \end{figure}
\end{frame}

\end{document}
